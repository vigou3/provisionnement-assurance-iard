%%% Copyright (C) 2019 Vincent Goulet, Frédérick Guillot, Mathieu Pigeon
%%%
%%% Ce fichier fait partie du projet
%%% «Provisionnement en assurance IARD»
%%% https://gitlab.com/vigou3/provisionnement-assurance-iard
%%%
%%% Cette création est mise à disposition sous licence
%%% Attribution-Partage dans les mêmes conditions 4.0
%%% International de Creative Commons.
%%% https://creativecommons.org/licenses/by-sa/4.0/

\documentclass[letterpaper,11pt,x11names,english,french]{memoir}
  \usepackage{natbib,url}
  \usepackage{babel}
  \usepackage[autolanguage]{numprint}
  \usepackage{amsmath,amsthm}
  \usepackage[noae]{Sweave}
  \usepackage{graphicx}
  \usepackage{currfile}                % pour noms fichiers de script
  \usepackage{framed}                  % env. snugshade*, oframed
  \usepackage[absolute]{textpos}       % éléments des pages de titre
  \usepackage[shortlabels]{enumitem}   % configuration listes
  \usepackage{relsize}                 % \smaller et al.
  \usepackage{manfnt}                  % \mantriangleright (puce)
  \usepackage{metalogo}                % \XeLaTeX logo
  \usepackage{fontawesome}             % icônes \fa*
  \usepackage{awesomebox}              % boites info, important, etc.
  \usepackage{answers}                 % exercices et solutions
  \usepackage{listings}                % code informatique
  \usepackage{pgf}                     % transparence pour couverture

  %% paquetage à supprimer?
  \usepackage{tikz}

  %%% =============================
  %%%  Informations de publication
  %%% =============================
  \title{Provisionnement en assurance IARD}
  \author{Vincent Goulet \and Frédérick Guillot \and Mathieu Pigeon}
  \renewcommand{\year}{2019}
  \renewcommand{\month}{0}
  \newcommand{\reposurl}{https://gitlab.com/vigou3/provisionnement-assurance-iard/}

  %%% ===================
  %%%  Style du document
  %%% ===================

  %% Polices de caractères
  \usepackage{fontspec}
  \usepackage{unicode-math}
  \defaultfontfeatures
  {
    Scale = 0.92
  }
  \setmainfont{Lucida Bright OT}
  [
    Ligatures = TeX,
    Numbers = OldStyle
  ]
  \setmathfont{Lucida Bright Math OT}
  \setmonofont{Lucida Grande Mono DK}
  \setsansfont{FiraSans}
  [
    Extension = .otf,
    UprightFont = *-Book,
    BoldFont = *-SemiBold,
    BoldItalicFont = *-SemiBoldItalic,
    Scale = 1.0,
    Numbers = OldStyle
  ]
  \newfontfamily\fullcaps{FiraSans}
  [
    Extension = .otf,
    UprightFont = *-Book,
    Scale = 1.0,
    Numbers = Uppercase
  ]
  \usepackage[babel=true]{microtype}
  \usepackage{icomma}

  %% Couleurs
  \usepackage{xcolor}
  \definecolor{comments}{rgb}{0.7,0,0}      % commentaires
  \definecolor{link}{rgb}{0,0.4,0.6}        % liens internes
  \definecolor{url}{rgb}{0.6,0,0}           % liens externes
  \definecolor{citation}{rgb}{0,0.5,0}      % citations
  \definecolor{codebg}{named}{LightYellow1} % fond code R
  \definecolor{lineno}{named}{gray}         % numéros de lignes

  %% Hyperliens
  \usepackage{hyperref}
  \hypersetup{%
    pdfauthor = \theauthor,
    pdftitle = \thetitle,
    colorlinks = true,
    linktocpage = true,
    urlcolor = {url},
    linkcolor = {link},
    citecolor = {citation},
    pdfpagemode = {UseOutlines},
    pdfstartview = {Fit}}
  \setlength{\XeTeXLinkMargin}{1pt}

  %% Affichage de la table des matières du PDF
  \usepackage{bookmark}
  \bookmarksetup{%
    open = true,
    depth = 3,
    numbered = true}

  %% Étiquettes de \autoref (redéfinitions compatibles avec babel).
  %% Attention! Les % à la fin des lignes sont importants sinon des
  %% blancs apparaissent dès que la commande \selectlanguage est
  %% utilisée... comme dans la bibliographie, par exemple.
  \addto\extrasfrench{%
    \def\appendixautorefname{annexe}%
    \def\figureautorefname{figure}%
    \def\exempleautorefname{exemple}%
    \def\exerciceautorefname{exercice}%
    \def\subfigureautorefname{figure}%
    \def\subsectionautorefname{section}%
    \def\subtableautorefname{tableau}%
    \def\tableautorefname{tableau}%
  }

  %% Table des matières (inspirée de classicthesis.sty)
  \renewcommand{\cftchapterleader}{\hspace{1.5em}}
  \renewcommand{\cftchapterafterpnum}{\cftparfillskip}
  \renewcommand{\cftsectionleader}{\hspace{1.5em}}
  \renewcommand{\cftsectionafterpnum}{\cftparfillskip}
  \renewcommand{\cfttableleader}{\hspace{1.5em}}
  \renewcommand{\cfttableafterpnum}{\cftparfillskip}
  \renewcommand{\cftfigureleader}{\hspace{1.5em}}
  \renewcommand{\cftfigureafterpnum}{\cftparfillskip}
  \setlength{\cftfigurenumwidth}{3.2em}

  %% Titres des chapitres
  \chapterstyle{hangnum}
  \renewcommand{\chaptitlefont}{\normalfont\Huge\sffamily\bfseries\raggedright}

  %% Marges, entêtes et pieds de page
  \setlength{\marginparsep}{7mm}
  \setlength{\marginparwidth}{13mm}
  \setlength{\headwidth}{\textwidth}
  \addtolength{\headwidth}{\marginparsep}
  \addtolength{\headwidth}{\marginparwidth}

  %% Titres des sections et sous-sections
  \setsecheadstyle{\normalfont\Large\sffamily\bfseries\raggedright}
  \setsubsecheadstyle{\normalfont\large\sffamily\bfseries\raggedright}
  \maxsecnumdepth{subsection}
  \setsecnumdepth{subsection}

  %% Listes. Paramétrage avec enumitem.
  \setlist[enumerate]{leftmargin=*,align=left}
  \setlist[enumerate,2]{label=\alph*),labelsep=*,leftmargin=1.5em}
  \setlist[enumerate,3]{label=\roman*),labelsep=*,leftmargin=1.5em,align=right}
  \setlist[itemize]{leftmargin=*,align=left}

  %% Options de babel
  \frenchbsetup{CompactItemize=false,%
    ThinSpaceInFrenchNumbers=true,
    ItemLabeli=\mantriangleright,
    ItemLabelii=\textendash,
    og=«, fg=»}
  \addto\captionsfrench{\def\figurename{{\scshape Fig.}}}
  \addto\captionsfrench{\def\tablename{{\scshape Tab.}}}
  \addto\captionsfrench{\def\listfigurename{Liste des figures}}

  %% Mise en forme du code source.
  %%
  %% Les numéros de lignes sont des hyperliens vers le point dans le
  %% document où l'on y fait référence dans une boite \gotorbox (voir
  %% plus loin).
  %%
  %% Pour y parvenir, j'utilise deux étiquettes pour une ligne: une
  %% basée sur un "nom" utilisé dans la rédaction, et une autre
  %% générée automatiquement à partir du numéro de chapitre et du
  %% numéro de ligne.
  %%
  %% Solution basée sur https://tex.stackexchange.com/q/191771
  \lstset{%
    language=R,
    extendedchars=true,
    basicstyle=\small\ttfamily\NoAutoSpacing,
    commentstyle=\color{comments}\slshape,
    keywordstyle=\mdseries,
    showstringspaces=false,
    numbers=left,
    numberstyle={%
      \color{lineno}\tiny\ttfamily%
      \ifnum\value{lstnumber}=\getrefnumber{code:\thechapter:\thelstnumber}%
        \renewcommand*\thelstnumber{\hyperlink{goto:\thechapter:\the\value{lstnumber}}{\bfseries\arabic{lstnumber}}}%
      \fi},
    firstnumber=\scriptfirstline,
    escapechar=`,
    moredelim=[il]{\#-*-}}

  %% Commandes pour créer les références et liens vers des numéros de
  %% lignes.
  \makeatletter
  \newcommand\labelline[1]{%
    \def\@currentlabel{\thelstnumber}%
    \label{lst:#1}\label{code:\thechapter:\the\value{lstnumber}}}
  \makeatother
  \newcommand{\reflines}[1]{%
    \hypertarget{goto:\thechapter:\getrefnumber{lst:#1}}{\ref{lst:#1}}--%
    \hypertarget{goto:\thechapter:\getrefnumber{lst:#1:fin}}{\ref{lst:#1:fin}}}

  %% L'entête des fichiers de script n'est pas affiché dans le
  %% document.
  \def\scriptfirstline{14}      % nombre magique!

  %%% =========================
  %%%  Nouveaux environnements
  %%% =========================

  %% Environnements d'exemples et al.
  \theoremstyle{plain}
  \newtheorem{thm}{Théorème}[chapter]
  \newtheorem{corrolaire}{Corollaire}[chapter]

  \theoremstyle{definition}
  \newtheorem{definition}{Définition}[chapter]
  \newtheorem{exemple}{Exemple}[chapter]

  %% Redéfinition de l'environnement titled-frame de framed.sty avec
  %% deux modifications: épaisseur des filets réduite de 2pt à 1pt;
  %% "(suite)" plutôt que "(cont)" dans la barre de titre
  %% lorsque l'encadré se poursuit après un saut de page.
  \renewenvironment{titled-frame}[1]{%
    \def\FrameCommand{\fboxsep8pt\fboxrule1pt
      \TitleBarFrame{\textbf{#1}}}%
    \def\FirstFrameCommand{\fboxsep8pt\fboxrule1pt
      \TitleBarFrame[$\blacktriangleright$]{\textbf{#1}}}%
    \def\MidFrameCommand{\fboxsep8pt\fboxrule1pt
      \TitleBarFrame[$\blacktriangleright$]{\textbf{#1\ (suite)}}}%
    \def\LastFrameCommand{\fboxsep8pt\fboxrule1pt
      \TitleBarFrame{\textbf{#1\ (suite)}}}%
    \MakeFramed{\advance\hsize-16pt \FrameRestore}}%
  {\endMakeFramed}

  %% Encadré générique avec titre basé sur titled-frame, ci-dessus.
  %% Sert pour les listes d'objectifs et les encadrés reliés aux
  %% problèmes (mises en situation) dans les chapitres. Arguments:
  %% couleur du cadre (optionnel; noir par défaut) et titre de la
  %% boite (obligatoire).
  \newenvironment{emphbox}[2][black]{%
    \colorlet{TFFrameColor}{#1}%
    \colorlet{TFTitleColor}{white}%
    \begin{titled-frame}{\sffamily #2}%
      \setlength{\parindent}{0pt}}%
    {\end{titled-frame}}

  %% Liste d'objectifs au début des chapitres
  \newenvironment{objectifs}{%
    \begin{emphbox}{\rule[-7pt]{0pt}{20pt} Objectifs du chapitre}
      \begin{itemize}[nosep]
        \small\sffamily}%
      {\end{itemize}\end{emphbox}}

  %% Problèmes (mises en situation) des chapitres: énoncé au début du
  %% chapitre; astuces en cours de chapitre; solution à la fin
  %% du chapitre.
  \newenvironment{prob-enonce}{%
    \begin{emphbox}[prob]{{\normalfont\faCogs}\; Énoncé du problème}}%
    {\end{emphbox}}
  \newenvironment{prob-astuce}{%
    \begin{emphbox}[prob]{{\normalfont\faBolt}\; Astuce}}%
    {\end{emphbox}}
  \newenvironment{prob-solution}{%
    \begin{emphbox}[prob]{{\normalfont\faLightbulbO}\; Solution du problème}}%
    {\end{emphbox}}

  %% Environnements de Sweave. Les environnements Sinput et Soutput
  %% utilisent Verbatim (de fancyvrb). On les réinitialise pour
  %% enlever la configuration par défaut de Sweave, puis on réduit
  %% l'écart entre les blocs Sinput et Soutput.
  \DefineVerbatimEnvironment{Sinput}{Verbatim}{}
  \DefineVerbatimEnvironment{Soutput}{Verbatim}{}
  \fvset{listparameters={\setlength{\topsep}{0pt}}}

  %% L'environnement Schunk est complètement redéfini en un hybride
  %% des environnements snugshade* et leftbar de framed.sty.
  \makeatletter
  \renewenvironment{Schunk}{%
    \setlength{\topsep}{1pt}
    \def\FrameCommand##1{\hskip\@totalleftmargin
       \vrule width 2pt\colorbox{codebg}{\hspace{3pt}##1}%
      % There is no \@totalrightmargin, so:
      \hskip-\linewidth \hskip-\@totalleftmargin \hskip\columnwidth}%
    \MakeFramed {\advance\hsize-\width
      \@totalleftmargin\z@ \linewidth\hsize
      \advance\labelsep\fboxsep
      \@setminipage}%
  }{\par\unskip\@minipagefalse\endMakeFramed}
  \makeatother

  %% Exercices et réponses
  \Newassociation{rep}{reponse}{reponses}
  \Newassociation{sol}{solution}{solutions}
  \newcounter{exercice}[chapter]
  \renewcommand{\theexercice}{\thechapter.\arabic{exercice}}
  \newenvironment{exercice}[1][]{%
    \begin{list}{}{%
        \refstepcounter{exercice}
        \ifthenelse{\equal{#1}{nosol}}{%
          \renewcommand{\makelabel}{\bfseries\theexercice}}{%
          \hypertarget{ex:\theexercice}{}
          \Writetofile{solutions}{\protect\hypertarget{sol:\theexercice}{}}
          \renewcommand{\makelabel}{%
            \bfseries\protect\hyperlink{sol:\theexercice}{\theexercice}}}
        \settowidth{\labelwidth}{\bfseries\theexercice}
        \setlength{\leftmargin}{\labelwidth}
        \addtolength{\leftmargin}{\labelsep}
        \setlist[enumerate,1]{label=\alph*),labelsep=*,leftmargin=1.5em}
        \setlist[enumerate,2]{label=\roman*),labelsep=0.5em,align=right}}
      \item}%
      {\end{list}}
  \renewenvironment{solution}[1]{%
    \begin{list}{}{%
        \renewcommand{\makelabel}{%
          \bfseries\protect\hyperlink{ex:#1}{#1}}
        \settowidth{\labelwidth}{\bfseries #1}
        \setlength{\leftmargin}{\labelwidth}
        \addtolength{\leftmargin}{\labelsep}
        \setlist[enumerate,1]{label=\alph*),labelsep=*,leftmargin=1.5em}
        \setlist[enumerate,2]{label=\roman*),labelsep=0.5em,align=right}}
      \item}
    {\end{list}}
  \renewenvironment{reponse}[1]{%
    \begin{enumerate}[label=\textbf{#1}]
    \item}%
    {\end{enumerate}}

  %% Listes de commandes
  \newenvironment{ttscript}[1]{%
    \begin{list}{}{%
        \setlength{\labelsep}{1.5ex}
        \settowidth{\labelwidth}{\fbox{#1}}
        \setlength{\leftmargin}{\labelwidth}
        \addtolength{\leftmargin}{\labelsep}
        \setlength{\parsep}{0.5ex plus0.2ex minus0.2ex}
        \setlength{\itemsep}{0.3ex}
        \renewcommand{\makelabel}[1]{\vphantom{|}##1\hfill}}}
    {\end{list}}

  %%% =====================
  %%%  Nouvelles commandes
  %%% =====================

  %% Noms de fonctions, code, etc.
  \newcommand{\code}[1]{\texttt{#1}}
  \newcommand{\pkg}[1]{\textbf{#1}}

  %% Hyperlien avec symbole de lien externe juste après; second
  %% argument peut être vide pour afficher l'url comme lien
  %% [https://tex.stackexchange.com/q/53068/24355 pour procédure de
  %% test du second paramètre vide]
  \newcommand{\link}[2]{%
    \def\param{#2}%
    \ifx\param\empty
      \href{#1}{\nolinkurl{#1}~\raisebox{-0.1ex}{\smaller\faExternalLink}}%
    \else
      \href{#1}{#2~\raisebox{-0.1ex}{\smaller\faExternalLink}}%
    \fi
  }

  %% Boite additionnelle (basée sur awesomebox.sty) pour changements
  %% au fil de la lecture.
  \newcommand{\gotorbox}[1]{%
    \awesomebox{\faMapSigns}{\aweboxrulewidth}{black}{#1}}

  %% Boite pour le nom du fichier de script correspondant au début des
  %% sections d'exemples.
  \newcommand{\scriptfile}[1]{%
    \begingroup
    \noindent
    \mbox{%
      \makebox[3mm][l]{\raisebox{-0.5pt}{\small\faChevronCircleDown}}\;%
      \smaller[1] {\sffamily Fichier d'accompagnement} {\ttfamily #1}}
    \endgroup}

  %% Lien vers GitLab dans la page de notices
  \newcommand{\viewsource}[1]{%
    \href{#1}{\faGitlab\ Voir sur GitLab}}

  %% Raccourcis usuels vg
  \newcommand{\pt}{{\scriptscriptstyle \Sigma}}
  \newcommand{\Esp}[1]{E\! \left[ #1 \right]}
  \newcommand{\esp}[1]{E [ #1 ]}
  \newcommand{\Var}[1]{\operatorname{Var}\! \left[ #1 \right]}
  \newcommand{\hVar}[1]{\widehat{\operatorname{Var}}\! \left[ #1 \right]}
  \newcommand{\var}[1]{\operatorname{Var} [ #1 ]}
  \newcommand{\hvar}[1]{\widehat{\operatorname{Var}} [ #1 ]}
  \newcommand{\Cov}{\operatorname{Cov}}
  \newcommand{\CV}{\operatorname{CV}}
  \ifxetex
    \newcommand{\mat}[1]{\symbf{#1}}
  \else
    \newcommand{\mat}[1]{\mathbf{#1}}
  \fi
  \ifxetex
    \newcommand{\Z}{\symbb{Z}}
  \else
    \newcommand{\Z}{\mathbb{Z}}
  \fi

  %%% =======
  %%%  Index
  %%% =======
  \newcommand{\bfhyperpage}[1]{\textbf{\hyperpage{#1}}}
  \renewcommand{\preindexhook}{%
    Les numéros de page en caractères gras indiquent les pages où les
    concepts sont introduits, définis ou expliqués.\vskip\onelineskip}
  \newcommand{\Index}[1]{\index{#1|bfhyperpage}}
  \newcommand{\indexcode}[1]{\index{#1@\code{#1}}}
  \newcommand{\Indexcode}[1]{\Index{#1@\code{#1}}}
  \newcommand{\icode}[1]{\indexcode{#1}\code{#1}}
  \newcommand{\Icode}[1]{\Indexcode{#1}\code{#1}}
  \makeindex

  %%% =======
  %%%  Varia
  %%% =======

  %%% Sous-figures
  \newsubfloat{figure}

  %%% Style de la bibliographie
  \bibliographystyle{francais}

  %% Aide pour la césure
  \hyphenation{%
  }

%  \includeonly{couverture-avant}

\begin{document}

\frontmatter

\pagestyle{empty}

%%% Copyright (C) 2019 Vincent Goulet, Frédérick Guillot, Mathieu Pigeon
%%%
%%% Ce fichier fait partie du projet
%%% «Provisionnement en assurance IARD»
%%% http://gitlab.com/vigou3/provisionnement-assurance-iard
%%%
%%% Cette création est mise à disposition sous licence
%%% Attribution-Partage dans les mêmes conditions 4.0
%%% International de Creative Commons.
%%% http://creativecommons.org/licenses/by-sa/4.0/

%%%
%%% Page de titre
%%%

\begingroup
\TPGrid{8}{64}
\textblockorigin{0mm}{0mm}
\setlength{\parindent}{0mm}
\setlength{\textwidth}{\paperwidth}
\addtolength{\textwidth}{-2\TPHorizModule}

\def\titlefmt{%
  \sffamily\bfseries\fontsize{42}{42}\selectfont\thetitle\par}
\def\authorsfmt{%
  \sffamily\mdseries\fontsize{28}{28}\selectfont
  V.\ Goulet \quad F.\ Guillot \quad M.\ Pigeon}
\def\affiliations{%
  \sffamily\mdseries
  \fontsize{28}{38}\selectfont
  Vincent Goulet \par
  \fontsize{16}{20}\selectfont
  École d'actuariat \\ Université Laval \par
  \fontsize{28}{40}\selectfont
  Frédérick Guillot \par
  \fontsize{16}{20}\selectfont
  Recherche et innovation \\ Co-operators \par
  \fontsize{28}{40}\selectfont
  Mathieu Pigeon \par
  \fontsize{16}{20}\selectfont
  Département de mathématiques \\ Université du Québec à Montréal}
\def\edition{%
  \sffamily\mdseries\fontsize{16}{16}\selectfont
  Édition {\fullcaps\year}.\month}

%% image de fond
\begin{textblock*}{\paperwidth}(0mm,0mm)
  \includegraphics[width=\paperwidth,%
                   keepaspectratio=true]{images/Sciurus_carolinensis}
\end{textblock*}

%% titre
\begin{textblock*}{0.7\textwidth}(\TPHorizModule,6\TPVertModule)
  \textcolor{black!10}{\titlefmt}
\end{textblock*}

%% auteurs
\begin{textblock*}{\textwidth}(\TPHorizModule,14.5\TPVertModule)
  \textcolor{black!10}{\authorsfmt}
\end{textblock*}

\null\cleardoublepage

%%%
%%% Page frontispice
%%%

%% titre
\begin{textblock*}{0.7\textwidth}(\TPHorizModule,6\TPVertModule)
  \titlefmt
\end{textblock*}

%% auteurs et affiliations
\begin{textblock*}{\textwidth}(\TPHorizModule,14.5\TPVertModule)
  \affiliations
\end{textblock*}

%% édition
\begin{textblock*}{\textwidth}(\TPHorizModule,58\TPVertModule)
  \edition
\end{textblock*}
\endgroup

%%% Local Variables:
%%% mode: latex
%%% TeX-master: "provisionnement-assurance-iard"
%%% TeX-engine: xetex
%%% coding: utf-8
%%% End:

\null\cleardoublepage           % cf. section 2.2 textpos.pdf

%%% Copyright (C) 2018 Vincent Goulet
%%%
%%% Ce fichier fait partie du projet
%%% «Provisionnement en assurance IARD»
%%% http://github.com/vigou3/provisionnement-assurance-iard
%%%
%%% Cette création est mise à disposition selon le contrat
%%% Attribution-Partage dans les mêmes conditions 4.0
%%% International de Creative Commons.
%%% http://creativecommons.org/licenses/by-sa/4.0/

\begingroup
\calccentering{\unitlength}
\begin{adjustwidth*}{\unitlength}{-\unitlength}
  \setlength{\parindent}{0pt}
  \setlength{\parskip}{\baselineskip}
  \small

  \raisebox{-2.5mm}{%
    \includegraphics[height=7mm,keepaspectratio=true]{by-sa}} %
  {\theauthor}, {\year}

  {\textcopyright} {\year} par {\theauthor}. «\thetitle» mis à
  disposition selon le contrat
  \href{http://creativecommons.org/licenses/by-sa/4.0/deed.fr}{%
    Attribution-Partage dans les mêmes conditions 4.0 International}
  de Creative Commons. En vertu de ce contrat, vous êtes libre de:
  \begin{itemize}
  \item \textbf{partager} --- reproduire, distribuer et communiquer
    l'{\oe}uvre;
  \item \textbf{remixer} --- adapter l'{\oe}uvre;
  \item utiliser cette {\oe}uvre à des fins commerciales.
  \end{itemize}
  Selon les conditions suivantes:

  \begin{tabularx}{\linewidth}{@{}lX@{}}
    \raisebox{-9mm}[0mm][13mm]{%
      \includegraphics[height=11mm,keepaspectratio=true]{by}} &
    \textbf{Attribution} --- Vous devez créditer l'{\oe}uvre, intégrer
    un lien vers le contrat et indiquer si des modifications ont été
    effectuées à l'{\oe}uvre. Vous devez indiquer ces informations par
    tous les moyens possibles, mais vous ne pouvez suggérer que
    l'Offrant vous soutient ou soutient la façon dont vous avez utilisé
    son {\oe}uvre. \\
    \raisebox{-9mm}{\includegraphics[height=11mm,keepaspectratio=true]{sa}}
    & \textbf{Partage dans les mêmes conditions} --- Dans le cas où vous
    modifiez, transformez ou créez à partir du matériel composant
    l'{\oe}uvre originale, vous devez diffuser l'{\oe}uvre modifiée dans
    les même conditions, c'est à dire avec le même contrat avec lequel
    l'{\oe}uvre originale a été diffusée.
  \end{tabularx}

  \textbf{Code source} \\
  \viewsource{\ghurl}

  \textbf{Couverture} \\
  Écureuil gris (\emph{Sciurus carolinensis}) photographié dans le
  comté d'Oxfordshire. Cette espèce d'écureuil
  originaire de l'est de l'Amérique du Nord a été introduite en
  Angleterre au début du vingtième siècle, où elle est maintenant
  considérée invasive.

  Crédit photo: {\textcopyright} Charlesjsharp,
  \href{https://creativecommons.org/licenses/by-sa/4.0/at/deed.fr}{CC
    BY-SA 4.0 International}, via
  \href{https://commons.wikimedia.org/w/index.php?curid=47900137}{Wikimedia
    Commons}.
\end{adjustwidth*}
\endgroup

%%% Local Variables:
%%% mode: latex
%%% TeX-engine: xetex
%%% TeX-master: "provisionnement-assurance-iard"
%%% coding: utf-8
%%% End:

\clearpage

\pagestyle{companion}

%\include{introduction}
\tableofcontents
\cleartorecto
\listoftables
\cleartorecto
\listoffigures

\mainmatter

%%% Copyright (C) 2019 Vincent Goulet, Frédérick Guillot, Mathieu Pigeon
%%%
%%% Ce fichier fait partie du projet
%%% «Provisionnement en assurance IARD»
%%% https://gitlab.com/vigou3/provisionnement-assurance-iard
%%%
%%% Cette création est mise à disposition sous licence
%%% Attribution-Partage dans les mêmes conditions 4.0
%%% International de Creative Commons.
%%% https://creativecommons.org/licenses/by-sa/4.0/

\chapter{Présentation générale}
\label{chap:presentation}

L'Autorité des marchés financiers (AMF) définit ainsi les provisions
et réserves en assurance IARD:
\begin{quote}
  Processus d'évaluation du montant total nécessaire pour
  acquitter tous les paiements futurs associés aux sinistres déjà
  survenus en date d'évaluation (ex. au 31 décembre).
\end{quote}

L'évaluation des provision pour sinistres constitue un exercice
crucial dans les opérations des assureurs de dommages. En effet,
provisions représentent environ 75~\% de leur passif. Si les
provisions sont sous-évaluées:
\begin{itemize}
\item la santé financière de la compagnie est surévaluée;
\item la compagnie s'expose au risque de défaut sur ses paiements
  futurs;
\item l'assureur est exposé à la ruine technique.
\end{itemize}
La situation inverse n'est pas préférable, car si les provisions sont
surévaluées, alors:
\begin{itemize}
\item les dépenses sont plus élevées;
\item le profit diminue;
\item les impôts diminuent;
\item le surplus diminue;
\item la valeur de la compagnie est moindre.
\end{itemize}

Un rôle important et central de l'actuaire en assurances IARD est le
calcul des provisions (ou des réserves). Il s'agit d'ailleurs de l'un
des deux rôles exclusivement réservé à un actuaire (fellow de l'ICA).
Les réserves ont pour objectif de permettre le règlement complet des
engagements pris par l'assureur envers ses assurés. Elles sont liées
au concept même d'assurance, généralement imposées par diverses
réglementations, pour tenir compte du cycle de production inversé que
comprend le marché des assurances. En effet, l'assureur ne peut
évaluer les coûts réels que représente un assuré qu'après l'expiration
du contrat et la fermeture de tous les dossiers liés, alors que le
montant de la prime doit être déterminé en début de contrat.

À cause de cette inversion du cycle, un contrôle rigoureux de la
solvabilité des compagnies est essentiel afin de protéger les assurés,
les actionnaires et/ou l'État. La législation des assurances impose
aux compagnies d'assurances IARD de garder en réserve un montant
suffisant afin de permettre le paiement de tous les sinistres
encourus. Un rôle important et central de l'actuaire de la compagnie
est de prévoir, avec le maximum de précision, le montant nécessaire de
cette réserve.

Le développement typique d'un sinistre est illustré à la
\autoref{fig:presentation:evolIN}. Le sinistre survient à la date de survenance
($t_1$) et est déclaré à l'assureur à la date de déclaration
($t_2$). Pour plusieurs situations (incendie, dommages matériels à une
voiture, etc.), ces deux dates correspondent (ou presque...) mais pour
d'autres situations (dommages corporels, responsabilité civile), une
période de temps plus ou moins longue peut séparer ces deux
moments. Par la suite, un ou plusieurs paiements peuvent être
effectués ($t_3$, $t_4$ et $t_5$) avant la fermeture du dossier
($t_6$).

\begin{figure}
  \centering
  %   \includegraphics[height=0.50\textwidth,
  %   width=0.99\textwidth]{ChapitreII/IndLossRes001}
% \begin{tikzpicture}[snake=zigzag, line before snake = 10mm, line
%      after snake = 10mm]
  \begin{tikzpicture}[scale=0.85]
    \draw[->,color=black,line width=1.5pt] (2,0) -- (15,0);

    \foreach \x in {3,4,8,9,10,14}   \draw (\x cm,2pt) -- (\x cm,-3pt);

    \draw (3.05,0) node[below=3pt] {$t_1$};
    \draw (4.05,0) node[below=3pt] {$t_2$};
    \draw (8.05,0) node[below=3pt] {$t_3$};
    \draw (9.05,0) node[below=3pt] {$t_4$};
    \draw (10.05,0) node[below=3pt] {$t_5$};
    \draw (14.05,0) node[below=3pt] {$t_6$};

    \draw[line width=1pt,|-|,color=black] (3,-1) to  (4,-1);
    \draw (3.5,-2.5) node[above=3pt,color=black] {IBNR};
    \draw[line width=1pt,|-|,color=gray] (4,-1.5) to  (14,-1.5);
    \draw (8.8,-2.5) node[above=3pt,color=gray] {RBNS};


    % \draw[line width=1pt,|-|,color=black] (3,-1) to  (4,-1);
    % \draw (3.5,-2.5) node[above=3pt,color=black] {IBNR};
    \draw[line width=1pt,|-|,color=black] (4,-3.3) to  (8,-3.3);
    \draw (6,-4.3) node[above=3pt,color=black] {RBNP};
    \draw[line width=1pt,|-|,color=black] (8,-2.8) to  (14,-2.8);
    \draw (11,-3.8) node[above=3pt,color=black] {RBNS};

    \draw[line width=1pt,-|,color=black] (2,-4.3) to  (3,-4.3);
    \draw (2.5,-5.3) node[above=3pt,color=black] {CBNI};

    \draw[line width=1pt,|-,color=black] (14,-4.3) to  (15,-4.3);
    \draw (14.5,-5.3) node[above=3pt,color=black] {S};

    \draw[->] (3,3) -- (3,0.5);
    \draw (3,3) node[above=3pt,color=black] {Survenance};
    \draw[->] (4,2) -- (4,0.5);
    \draw (4.15,2) node[above=3pt,color=black] {Déclaration};
    \draw[->] (8,3) -- (8,0.5);
    \draw[->] (9,3) -- (9,0.5);
    \draw (9,3) node[above=3pt,color=black] {Paiements};
    \draw[->] (10,3) -- (10,0.5);
    \draw[->] (14,3) -- (14,0.5);
    \draw (14,3) node[above=3pt,color=black] {Fermeture};
  \end{tikzpicture}
  \caption{Évolution d'un sinistre}
  \label{fig:presentation:evolIN}
\end{figure}

\begin{figure}
  \setlength{\unitlength}{2mm}
  \begin{picture}(65,12)
    \setlength{\fboxsep}{1.5pt}
    \put(0,0){\colorbox{lightgray}{\makebox(12.5,1.5){}}}
    \thicklines
    \put(0,0.75){\vector(1,0){65}}
    \put( 6,0){\line(0,1){1.5}}
    \put(15,0){\line(0,1){1.5}}
    \put(22,0){\line(0,1){1.5}}
    \put(24,0){\line(0,1){1.5}}
    \put(27,0){\line(0,1){1.5}}
    \put(35,0){\line(0,1){1.5}}
    \put(45,0){\line(0,1){1.5}}
    \put(51,0){\line(0,1){1.5}}
    \put(54,0){\line(0,1){1.5}}
    \put(60,0){\line(0,1){1.5}}

    \thinlines
    \put( 6,9){\vector(0,-1){6}}
    \put(15,9){\vector(0,-1){6}}
    \put(22,7){\vector(0,-1){4}}
    \put(24,7){\vector(0,-1){4}}
    \put(27,7){\vector(0,-1){4}}
    \put(35,9){\vector(0,-1){6}}
    \put(45,9){\vector(0,-1){6}}
    \put(51,7){\vector(0,-1){4}}
    \put(54,7){\vector(0,-1){4}}
    \put(60,9){\vector(0,-1){6}}

    \small
    \put(0.1,-1.5){\makebox(12.5,0){police en vigueur}}
    \put( 6,10.5){\makebox(0,0){sinistre}}
    \put(15,10.5){\makebox(0,0){déclaration}}
    \put(24.5,8.5){\makebox(0,0){paiements}}
    \put(35,10.5){\makebox(0,0){fermeture}}
    \put(45,10.5){\makebox(0,0){réouverture}}
    \put(52.5,8.5){\makebox(0,0){paiements}}
    \put(60,10.5){\makebox(0,0){fermeture}}
  \end{picture}
  \caption{Évolution d'un sinistre (alternative)}
  \label{fig:presentation:evolIN:alt}
\end{figure}

À une certaine date d'évaluation (par exemple, le 31 décembre), les
sinistres peuvent être séparés en plusieurs catégories en fonction du
stade atteint par leur développement:
\begin{itemize}
\item si la date d'évaluation se trouve entre la date de survenance et
  la date de déclaration, le sinistre est considéré comme
  \emph{Incurred But Not Reported} (IBNR)\footnote{%
    Dans le cadre du cours, on conservera les noms anglais pour les
    catégories.}; %
\item si la date d'évaluation se trouve entre la date de déclaration
  et la date du premier paiement, le sinistre est considéré comme
  \emph{Reported But Not Paid} (RBNP)\footnote{%
    Cette catégorie est parfois regroupée avec la catégorie
    suivante.}; %
\item si la date d'évaluation se trouve entre la date du premier
  paiement et la date de fermeture du dossier, le sinistre est
  considéré comme \emph{Reported But Not Settled} (RBNS);
\item si la date d'évaluation se trouve avant la date de survenance,
  alors le sinistre est considéré comme \emph{Covered But Not
    Incurred} claims (CBNI); et
\item si la date d'évaluation se trouve après la date de fermeture du
  dossier, le sinistre est considéré comme \emph{Settled} (ou
  \emph{closed}) (S).
\end{itemize}
Cette classification est illustrée à la \autoref{fig:presentation:E1E2}.

\begin{figure}
  \centering
  \includegraphics[height=0.45\textwidth, width=0.45\textwidth]{images/E1a}
  \includegraphics[height=0.45\textwidth, width=0.45\textwidth]{images/E2a}
  \caption{Exemples de classification des sinistres. Le graphique de
    gauche présente l'évolution de $10$ sinistres, les tailles des
    points étant proportionnelles aux montants payés. En supposant que
    l'évaluation a lieu le 1 janvier 2003, le graphique de droite
    présente la classification des sinistres: les sinistres
    1,3,4,5,6,8 et 10 sont S, le sinistre 2 est RBNS, le sinistre 7
    est IBNR et le sinistre 9 est RBNP.}
  \label{fig:presentation:E1E2}
\end{figure}

En plus du développement que l'on vient de décrire, une compagnie
d'assurance détient généralement une certaine expérience via ses
experts. Cette expérience se traduira par des prédictions du montant
total d'un sinistre réalisées à différents moments entre la date de
déclaration et la date de fermeture d'un dossier. Ce processus est
illustré à la \autoref{fig:presentation:evolRES}. Il est à noter qu'un
changement dans le montant prédit de réserve n'accompagne pas
nécessairement un paiement.

Il faut alors chercher à construire un modèle permettant de
représenter l'évolution des sinistres et à estimer les paramètres de
ce modèle à partir de l'information disponible.

\begin{figure}
  \centering
  \setlength{\unitlength}{5mm}
  \small
  \begin{picture}(22,10)
    \put(1,5){\vector(1,0){21}}
    \put(1,3){\vector(1,0){21}}
    \put(2,9){\vector(0,-1){3}}
    \put(4,7){\vector(0,-1){1}}
    \put(9,9){\vector(0,-1){3}}
    \put(10,9){\vector(0,-1){3}}
    \put(11,9){\vector(0,-1){3}}
    \put(20,9){\vector(0,-1){3}}
    \put(2,4.75){\line(0,1){0.5}}
    \put(4,4.75){\line(0,1){0.5}}
    \put(9,4.75){\line(0,1){0.5}}
    \put(10,4.75){\line(0,1){0.5}}
    \put(11,4.75){\line(0,1){0.5}}
    \put(20,4.75){\line(0,1){0.5}}
    % \put(2,2.75){\line(0,1){0.5}}
    \put(4,2.75){\line(0,1){0.5}}
    \put(9,2.75){\line(0,1){0.5}}
    \put(10,2.75){\line(0,1){0.5}}
    % \put(11,2.75){\line(0,1){0.5}}
    \put(20,2.75){\line(0,1){0.5}}
    \put(15,2.75){\line(0,1){0.5}}

    \put(2,4){$t_1$} \put(4,4){$t_2$}
    \put(9,4){$t_3$} \put(10,4){$t_4$}
    \put(11,4){$t_5$}
    \put(15,4){$t_{5.5}$}
    \put(20,4){$t_6$}
    \put(0.5,10){Survenance}
    \put(2.5,8){Déclaration}
    \put(8,10){Paiements}
    \put(19,10){Fermeture}

    \put(3,0){Réserve}
    \put(3,-1){initiale}
    \put(11,0){Ajustements}
    \put(19,0){Fermeture}
    \put(4,1){\vector(0,1){1}}
    \put(9,1){\vector(0,1){1}}
    \put(10,1){\vector(0,1){1}}
    \put(15,1){\vector(0,1){1}}
    \put(20,1){\vector(0,1){1}}

    % \put(2,-1.5){\textcolor{rougeG}{\line(1,0){2}}}
    % \put(2,-1.75){\textcolor{rougeG}{\line(0,1){0.5}}}
    % \put(4,-1.75){\textcolor{rougeG}{\line(0,1){0.5}}}
    % \put(2,-2.75){\textcolor{rougeG}{IBNR}}

    % \put(4,-2.5){\textcolor{bleuG}{\line(1,0){5}}}
    % \put(4,-2.75){\textcolor{bleuG}{\line(0,1){0.5}}}
    % \put(9,-2.75){\textcolor{bleuG}{\line(0,1){0.5}}}
    % \put(6,-3.75){\textcolor{bleuG}{RBNP}}
    % \put(9,-1.5){\textcolor{green}{\line(1,0){11}}}
    % \put(9,-1.75){\textcolor{green}{\line(0,1){0.5}}}
    % \put(20,-1.75){\textcolor{green}{\line(0,1){0.5}}}
    % \put(14,-2.75){\textcolor{green}{RBNS}}

    % \put(3, 1){\circle*{.5}}
  \end{picture}
  \caption{Évaluations des experts}
  \label{fig:presentation:evolRES}
\end{figure}


\section{Approches collectives et individuelles}
\label{sec:presentation:approches}

L'estimation des paramètres d'un modèle décrivant en détail le
développement individuel des sinistres (voir la
\autoref{fig:presentation:evolIN}) demande une base de données
détaillée (dates exactes, montant de chacun des paiements, etc.) et
fiable, de même que des moyens calculatoires (informatiques)
importants. De telles ressources n'étant disponibles que depuis peu
(fin des années 90 environ), plusieurs modèles ont été développés pour
représenter la \emph{dynamique collective} des sinistres\footnote{%
  De nombreuses autres raisons peuvent motiver l'utilisation
  d'approches collectives: la simplicité et la robustesse des modèles
  collectifs, le risque de sur-paramétrisation des approches
  individuelles, etc. Une comparaison des approches collectives et
  individuelles est réalisée dans \cite{Jin2014}.}. %

\subsection{Approches collectives}
\label{sec:presentation:approches:collectives}

Les modèles collectifs pour les réserves ont été étudiés
principalement depuis le début des années 80. Parmi ces modèles, le
modèle \emph{Chain--Ladder} (ou modèle de Mack dans sa version
stochastique, voir \cite{Mack93}) occupe une position particulière
étant à la base de la plupart des modèles utilisés en pratique. Ce
modèle (de même que ses nombreuses extensions et variantes) est
construit à partir d'une base de données résumées par période
(typiquement une année) de survenance et par période de développement
en un tableau nommé \emph{triangle de développement} (voir
\autoref{sec:presentation:triangles}). Ces modèles seront étudiés dans
les deux prochains chapitres.

À partir de cette même structure, ou à partir d'une version
\emph{incrémentale} de cette dernière, plusieurs modèles paramétriques
ont également été proposés (voir \citet{Hertig}, \citet{RV98},
\citet{EV2005} et \citet{Taylorbook}). Ces modèles s'inscrivent plus
ou moins directement dans la théorie des modèles linéaires généralisés
(\emph{Generalized Linear Models}) et des modèles linéaires
généralisés mixtes (\emph{Generalized Linear Mixed Models}) et seront
brièvement introduits au Chapitre 9. Un lecteur intéressé par un
historique plus complet des modèles collectifs est invité à consulter
\citet{WuthrichBook} et \citet{Engl02}.

\subsection{Approches individuelles}
\label{sec:presentation:approches:individuelles}

Très récemment, des modèles individuels ont commencé à faire leur
apparition dans la littérature. Les articles \citet{Arjas} et
\citet{Norberg, Norberg99} ont proposé une structure stochastique
individuelle en temps continu pour les sinistres et l'évaluation des
réserves. À partir de cette base, quelques modèles ont été développés,
par exemple \citet{Haastrup}, \citet{Lars07}, \citet{Zhao09},
\citet{ZhaoIME2} et \citet{AntonioPlat}. À l'exception notable de ce
dernier article, aucune mise en pratique réaliste de ces modèles n'a
été réalisée. Dans une autre voie, des modèles individuels basés sur
la structure \emph{chain--ladder} ont été proposés (voir
\citet{PigAntDen2013} et \citet{PigAntDen2014}). Enfin, quelques
modèles non-paramétriques ont également été proposés (voir
\citet{Drieskens} et \citet{Rosenlund} par exemple). Les approches
individuelles ne seront pas traitées dans le cadre de ce cours.

\begin{exemple}
  On considère un cas d'assurance responsabilité civile automobile
  (blessure corporelle):
  \begin{itemize}
  \item le 15 novembre 2004, un assuré frappe un piéton avec sa
    voiture;
  \item le 10 janvier 2005, le piéton commence à ressentir de violents
    maux de dos et de tête, conséquences de l'accident. Une poursuite
    est engagée contre le conducteur par le piéton;
  \item le 22 janvier 2005, l'assuré contacte son assureur pour
    l'avertir de la réclamation et le service d'indemnisation de la
    compagnie d'assurance évalue automatiquement la réclamation à
    \nombre{20 000}~\$;
  \item le 1er mars 2005, des spécialistes de la compagnie d'assurance
    évalue la réclamation à \nombre{200 000}~\$;
  \item le 30 mars 2005, le piéton refuse l'offre de \nombre{180
      000}~\$ de la compagnie d'assurance et va en cour;
  \item le 30 juin 2005, la compagnie d'assurance paie \nombre{15
      000}~\$ de frais légaux pour les services d'avocat (pour cette
    cause);
  \item le 30 mai 2006, la compagnie d'assurance paie \nombre{30
      000}~\$ de frais légaux pour les services d'avocat (pour cette
    cause);
  \item Le 6 octobre 2007, la décision de la cour est une indemnité de
    \nombre{250000}~\$.
  \end{itemize}
  Illustrer par une ligne du temps la vie du sinistre.

  Dessin à faire...

  \textbf{Remarques: }
  \begin{itemize}
  \item le 31 décembre 2004, la compagnie se devait d'inscrire une
    réserve pour l'accident du 15 novembre;
  \item aucune réclamation n'avait pourtant été soumise;
  \item ce genre de provisions s'appelle IBNR (\emph{Incurred But Not
      Reported});
  \item des paiements ont lieu tout au long de la vie de la
    réclamation;
  \item le 31 décembre 2005 et 2006, la compagnie se devait d'inscrire
    pour l'année d'accident 2004, une réserve pour ce sinistre;
  \item le 31 décembre 2007, malgré qu'il semble que le sinistre soit
    clos, il existe peut-être une possibilité de réouverture du
    dossier...
  \end{itemize}
  \qed
\end{exemple}

Lors de l'estimation des réserves, certaines hypothèses sont utilisées:
\begin{itemize}
\item les techniques d'évaluation des réserves seront utilisées avec
  des dollars constants; et
\item aucune actualisation des montants n'est considérée.
\end{itemize}

Pendant longtemps, l'actualisation des réserves a été illégale dans
l'évaluation du passif des polices et ce, même si l'effet du facteur
d'escompte n'est pas négligeable sur des branches d'assurance très
longues.

%\begin{definition}[Réserve individuelle de sinistre]
%Il s'agit de l'estimation, réalisée par les
%spécialistes de la compagnie d'assurance, du montant total qu'il reste à payer pour u%n sinistre. Cette estimation est
%réalisée au moment de l'évaluation (à l'instant présent).
%\end{definition}

%\begin{definition}[Sinistres survenus mais non-rapportés (\textit{IBNR})]
%Il s'agit de sinistres (ou de montants de sinistres) qui sont survenus
%mais dont l'assureur ne connaît pas encore l'existence.
%\end{definition}

%Les pertes totales, ou encore l'encouru total, pour un
%assureur correspond à tout ce qu'il a payé et tout ce qu'il lui reste
%à payer (réserve).  De manière plus rigoureuse, on peut
%définir les éléments d'une réserve.

%\begin{definition}
%Une réserve est composée des éléments suivants:
%\begin{itemize}
%\item les réserves individuelles;
%\item les ajustements à venir des réserves individuelles;
%\item les sinistres survenus mais non-rapportés (IBNR);
%\item les réclamations fermées qui peuvent rouvrir; et
%\item les sinistres rapportés mais non-enregistrés.
%\end{itemize}
%\end{definition}

La dynamique de la vie des sinistres dépend beaucoup de la ligne
d'affaires. Pour un type de risque particulier, les sinistres sont
constatés (avec plus ou moins de retard), puis ensuite payés, encore
avec plus ou moins de retard.

Pour fin d'illustration, si un accident a lieu durant l'année $T$, le
\autoref{tab:presentation:plo} donne une idée des cadences de
règlement pour différentes branches d'assurance.

\begin{table}
  \centering
  \caption{Cadences de règlement pour différentes branches
    d'assurance}
  \label{tab:presentation:plo}
  \begin{tabular}{llllll}
    \toprule
    Règlements à la fin de l'année  & $T$ & $T+1$ & $T+2$ & $T+3$ & $T+4$\\
    \midrule
    Multirisque habitation & $55\%$ & $90\%$ & $94\%$ & $95\%$ & $96\%$ \\
    Automobile & $55\%$ & $79\%$ & $84\%$ & $89\%$ & $90\%$ \\
    \textit{dont corporel} & $13\%$ & $38\%$ & $50\%$ & $65\%$ & $72\%$ \\
    Responsabilité civile & $10\%$ & $25\%$ & $35\%$ & $40\%$ & $45\%$ \\
    \bottomrule
  \end{tabular}
\end{table}

Le but principal des deux chapitres à venir est de développer des
techniques afin d'estimer le plus précisément possible le montant des
réserves pour chacun des bilans financiers de l'assureur, c'est-à-dire
à la fin des années $T$, $T+1$, etc. Ainsi, on peut établir l'équation
suivante:
\begin{align*}
  \text{Réserve estimée}
  &= \text{Pertes ultimes estimées} - \text{montants payés}\\
  % &= (\text{montants payés} + \text{Réserves individuelles} + \text{IBNR} + \ldots)  - %\text{montants payés}\\
  &= \text{Réserves RBNS} + \text{Réserves IBNR} \\
  &\phantom{=} + \text{Réclamations fermées qui peuvent rouvrir}.
\end{align*}


\section{Techniques intuitives}
\label{sec:presentation:techniques}

Dans certaines situations où les paiements futurs sont très stables et
très prévisibles, il est possible d'utiliser des méthodes simples pour
l'estimation des réserves.

\begin{description}
\item[Méthode des réserves enregistrées] L'idée de la méthode est
  d'utiliser le total des estimations faites par les experts de la
  compagnie (les réserves enregistrées) et d'ajouter un pourcentage
  arbitraire afin d'inclure l'incertitude quant à l'évolution possible
  des coûts futurs. Le problème majeur de cette méthode est d'estimer
  correctement ce pourcentage.

\item[Méthode des rapports sinistres/primes espérés] L'idée de la
  méthode est d'utiliser le rapport sinistres/primes prédit (attendu)
  de la branche d'affaire et les montants payés à ce jour de
  réclamations afin d'estimer la réserve. Ainsi, pour une année $i$
  fixée, on a
  \begin{align*}
    \text{Pertes ultimes estimées}_i
    &= \left(\text{Rapport sinistres/primes espéré}\right)_i \times \text{Primes acquises}_i\\
    \text{Réserve estimée}_i
    &= \text{Pertes ultimes estimées}_i - \text{Montants payés}_i
  \end{align*}
  En sommant toutes les années de couverture, on obtient une
  estimation de la réserve:
  \begin{align*}
    \text{Réserve estimée totale} &= \sum_i \text{Réserve estimée}_i.
  \end{align*}
\end{description}

\begin{exemple}
  On considère les primes acquises et les montants totaux payés à la
  fin de 2005, présentés dans le tableau ci-dessous, pour les années
  2000 à 2005:
  \begin{center}
    \begin{tabular}{|c | c c|}\hline
      Année & Primes acquises & Montants payés \\ \hline
      2000 & $\nombre{100000}$ & $\nombre{58000}$\\
      2001 & $\nombre{105000}$ & $\nombre{50000}$\\
      2002 & $\nombre{110000}$ & $\nombre{45000}$\\
      2003 & $\nombre{112500}$ & $\nombre{40000}$\\
      2004 & $\nombre{120000}$ & $\nombre{25000}$\\
      2005 & $\nombre{115000}$ & $\nombre{12000}$ \\ \hline
    \end{tabular}
  \end{center}

  Estimer la réserve totale pour les années 2000 à 2005 si on prévoit
  un rapport sinistres/primes de $60\%$ pour toutes les années.

  \begin{align*}
    \text{Primes acquises}_{total}
    &= \nombre{662500}\\
    \text{Pertes ultimes estimées}_{total}
    &= \left(\text{Rapport sinistres/primes espéré}\right)_{total} \times \text{Primes acquises}_{total}\\
    &= 0,60 \times \nombre{662500} = \nombre{397500}\\
    \text{Réserve estimée}_{total}
    &= \text{Pertes ultimes estimées}_{total} - \text{montants payés}_{total}\\
    &=  \nombre{397500} - \nombre{230000} = \nombre{167500}.
  \end{align*}
  Cette méthode souffre du problème qu'il est plus que probable que
  les pertes soient différentes de celles prévues (catastrophe,
  etc.). %
  \qed
\end{exemple}


\section{Triangles de développement}
\label{sec:presentation:triangles}

Afin d'étudier l'évolution des coûts au cours du temps et de pouvoir
estimer les réserves d'une compagnie d'assurance, on considère
généralement les données sous la forme d'un triangle de développement.
Ce dernier reflète la dynamique collective des sinistres.

La notation usuelle est la suivante:
\begin{itemize}
\item $i$ correspond à l'indice des années de survenance
  $i=1,\ldots,n$;
\item $j$ correspond à l'indice des années de développement
  $j=1,\ldots,n$;
\item $Y_{i,j}$ correspond au montant des sinistres survenus l'année
  $i$ et payés l'années $i+j$ (ou après $j$ années de développement).
  Pour cette variable, on parle aussi d'incréments; et
\item $C_{i,j}$ correspond aux paiements agrégés des sinistres
  survenus l'année $i$, en $j$ années de développement, i.e.
  $C_{i,j} = Y_{i,1} + Y_{i,2}+\ldots+Y_{i,j}$. Pour cette variable,
  on parle aussi de montants cumulés.
\end{itemize}

La sinistralité d'une branche peut ainsi être représentée par
\begin{center}
  \begin{tabular}{*{8}{c}}
    \toprule
    Année & \multicolumn{7}{c}{Développement (âge)} \\
    accident & $1$ & $2$ & $\cdots$ & $j$ & $\cdots$ & $J - 1$ & $J$ \\
    \midrule
    $1$ & $C_{1, 1}$ & $C_{1, 2}$ & $\cdots$ & $C_{1, j}$ & $\cdots$ & $C_{1, J-1}$ & $C_{1, J}$ \\
    $2$ & $C_{2, 1}$ & $C_{2, 2}$ & $\cdots$ & $C_{2, j}$ & $\cdots$ & $C_{2, J-1}$ \\
    $\vdots$ \\
    $i$ & $C_{i, 1}$ & $C_{i, 2}$ & $\cdots$ & $C_{i, j}$ \\
    $\vdots$ \\
    $I - 1$ & $C_{I-1, 1}$ & $C_{I-1, 2}$ \\
    $I$ & $C_{I, 1}$ \\
    \bottomrule
  \end{tabular}
\end{center}
ou
\begin{center}
  \begin{tabular}{*{8}{c}}
    \toprule
    Année & \multicolumn{7}{c}{Développement (âge)} \\
    accident & $1$ & $2$ & $\cdots$ & $j$ & $\cdots$ & $J - 1$ & $J$ \\
    \midrule
    $1$ & $Y_{1, 1}$ & $Y_{1, 2}$ & $\cdots$ & $Y_{1, j}$ & $\cdots$ & $Y_{1, J-1}$ & $Y_{1, J}$ \\
    $2$ & $Y_{2, 1}$ & $Y_{2, 2}$ & $\cdots$ & $Y_{2, j}$ & $\cdots$ & $Y_{2, J-1}$ \\
    $\vdots$ \\
    $i$ & $Y_{i, 1}$ & $Y_{i, 2}$ & $\cdots$ & $Y_{i, j}$ \\
    $\vdots$ \\
    $I - 1$ & $Y_{I-1, 1}$ & $Y_{I-1, 2}$ \\
    $I$ & $Y_{I, 1}$ \\
    \bottomrule
  \end{tabular}
\end{center}

La lecture des triangles de développement peut se faire selon divers
angles:
\begin{itemize}
\item la lecture par \textbf{colonne} correspond à l'année de
  développement $j$, et reflète l'évolution de la vie des sinistres;
\item la lecture par \textbf{ligne} correspond à l'année de survenance
  $i$, et reflète les changements de souscriptions, de taille du
  portefeuille, etc.; et
\item la lecture par \textbf{diagonale} correspond à l'année de
  calendrier.
\end{itemize}

\begin{exemple}
  On considère le triangle de paiements suivant (pour aider à la
  notation, on suppose que la première année est l'année 1997):
  \begin{center}
    \begin{tabular}{*{6}{c}}
      \toprule
      Année & \multicolumn{5}{c}{Développement (âge)} \\
      accident& 12 mois & 24 mois & 36 mois & 48 mois & 60 mois \\
      \midrule
      1997 & $\nombre{26312}$ & $\nombre{31467}$ & $\nombre{24672}$ & $\nombre{13055}$ & $\nombre{6158}$ \\
      1998 & $\nombre{30470}$ & $\nombre{35012}$ & $\nombre{25491}$ & $\nombre{12589}$ \\
      1999 & $\nombre{49756}$ & $\nombre{51831}$ & $\nombre{35267}$ \\
      2000 & $\nombre{50420}$ & $\nombre{52315}$ \\
      2001 & $\nombre{56762}$ \\
      \bottomrule
    \end{tabular}
  \end{center}
  Calculer les éléments suivants:

  \begin{itemize}
  \item les montants payés en 1999 pour les sinistres survenus pendant
    l'année 1999;
  \item les montants payés en 1999 pour les sinistres survenus pendant
    l'année 1997; et
  \item le montant total des sinistres payés pour l'année de
    survenance 1998 observé à la fin de l'année 2000.
  \end{itemize}

  On obtient
  \begin{itemize}
  \item En 1999, $\nombre{49756}\$$ ont été payés pour les sinistres
    survenus à l'année 1999;
  \item $\nombre{24672}\$$ ont été payés pour des sinistres survenus
    en 1997; et
  \item le montant total des sinistres payés pour l'année de
    survenance 1998 était, à la fin de 2000, de
    $\nombre{90973} = \nombre{30470} + \nombre{35012} +
    \nombre{25491}$.
  \end{itemize}
  \qed
\end{exemple}

Puisque l'idée du provisionnement est de \textbf{prévoir le montant
  final des sinistres} afin de provisionner pour les paiements
non-effectués, sous l'hypothèse que tous les sinistres seront réglés
après $n$ années, on cherche à compléter le tableau en remplissant le
triangle inférieur droit
\begin{center}
  \begin{tabular}{*{8}{c}}
    \toprule
    Année & \multicolumn{7}{c}{Développement (âge)} \\
    accident & $1$ & $2$ & $\cdots$ & $j$ & $\cdots$ & $J - 1$ & $J$ \\
    \midrule
    $1$ & $C_{1, 1}$ & $C_{1, 2}$ & $\cdots$ & $C_{1, j}$ & $\cdots$ & $C_{1, J-1}$ & $C_{1, J}$ \\
    $2$ & $C_{2, 1}$ & $C_{2, 2}$ & $\cdots$ & $C_{2, j}$ & $\cdots$ & $C_{2, J-1}$ & $\hat{C}_{1, J}$ \\
    $\vdots$ \\
    $i$ & $C_{i, 1}$ & $C_{i, 2}$ & $\cdots$ & $C_{i, j}$ & $\cdots$ & $\hat{C}_{i, J-1}$ & $\hat{C}_{1, J}$ \\
    $\vdots$ \\
    $I - 1$ & $C_{I-1, 1}$ & $C_{I-1, 2}$  & $\cdots$ & $\hat{C}_{i, j}$ & $\cdots$ & $\hat{C}_{i, J-1}$ & $\hat{C}_{1, J}$ \\
    $I$ & $C_{I, 1}$ & $\hat{C}_{I-1, 2}$  & $\cdots$ & $\hat{C}_{i, j}$ & $\cdots$ & $\hat{C}_{i, J-1}$ & $\hat{C}_{1, J}$ \\
    \bottomrule
  \end{tabular}
\end{center}
ou
\begin{center}
  \begin{tabular}{*{8}{c}}
    \toprule
    Année & \multicolumn{7}{c}{Développement (âge)} \\
    accident & $1$ & $2$ & $\cdots$ & $j$ & $\cdots$ & $J - 1$ & $J$ \\
    \midrule
    $1$ & $Y_{1, 1}$ & $Y_{1, 2}$ & $\cdots$ & $Y_{1, j}$ & $\cdots$ & $Y_{1, J-1}$ & $Y_{1, J}$ \\
    $2$ & $Y_{2, 1}$ & $Y_{2, 2}$ & $\cdots$ & $Y_{2, j}$ & $\cdots$ & $Y_{2, J-1}$ & $\hat{C}_{1, J}$ \\
    $\vdots$ \\
    $i$ & $Y_{i, 1}$ & $Y_{i, 2}$ & $\cdots$ & $Y_{i, j}$ & $\cdots$ & $\hat{C}_{i, J-1}$ & $\hat{C}_{1, J}$ \\
    $\vdots$ \\
    $I - 1$ & $Y_{I-1, 1}$ & $Y_{I-1, 2}$  & $\cdots$ & $\hat{C}_{i, j}$ & $\cdots$ & $\hat{C}_{i, J-1}$ & $\hat{C}_{1, J}$ \\
    $I$ & $Y_{I, 1}$ & $\hat{C}_{I-1, 2}$  & $\cdots$ & $\hat{C}_{i, j}$ & $\cdots$ & $\hat{C}_{i, J-1}$ & $\hat{C}_{1, J}$ \\
    \bottomrule
  \end{tabular}
\end{center}

Ayant estimé ces paiements futurs, le montant des provisions pour
l'année de survenance $i$ est alors donné par
\begin{align*}
  \hat{R}_i &= \hat{C}_{i,n} - C_{i, n+1-i} =
                  \hat{Y}_{i,n+2-i} + \hat{Y}_{i,n+3-i} + \dots +
                  \hat{Y}_{i,n}
\end{align*}
et le montant total des réserves est donné par
\begin{align*}
R &= \sum_{i=1}^n \hat{R}_i = \sum_{i=1}^n \hat{C}_{i,n} -
C_{i, n+1-i} = \sum_{(i,j) \in \Delta_n } \hat{Y}_{i,j},
\end{align*}
où $\Delta_n$ désigne l'ensemble
$\{(i,j), i+j \ge n+2, i \le n \text{ et } j \le n \}$.

Les modèles basés sur les triangles de développement supposent souvent
que tous les sinistres sont réglés après $n$ années. Afin d'assouplir
cette condition, il est aussi possible de considérer des ajustement
pour tenir compte des futures évolutions.


\section{Exercices}
\label{sec:presentation:exercices}

\Opensolutionfile{reponses}[reponses-presentation]
\Opensolutionfile{solutions}[solutions-presentation]

\begin{Filesave}{reponses}
\bigskip
\section*{Réponses}

\end{Filesave}

\begin{Filesave}{solutions}
\section*{Chapitre \ref*{chap:presentation}}
\addcontentsline{toc}{section}{Chapitre \protect\ref*{chap:presentation}}

\end{Filesave}

\begin{exercice}
 Le \autoref{tab:presentation:loss12} présente les différents paiements
 réalisés par l'assureur YTR.
  \begin{table}[!h]
    \centering
    \caption{Paiements réalisés par l'assureur YTR}
    \label{tab:presentation:loss12}
    \begin{tabular}{cccc}
      \toprule
      Date paiement & Numéro assuré & Date survenance & Montant\\
      \midrule
      04/2000 & 456 & 03/2000 & $200$\\
      09/2000 & 476 & 08/2000 & $225$\\
      02/2001 & 456 & 03/2000 & $40$\\
      10/2001 & 476 & 08/2000 & $57$\\
      01/2002 & 456 & 03/2000 & $90$\\
      04/2003 & 476 & 08/2000 & $102$\\
      02/2004 & 476 & 08/2000 & $16$\\
      10/2001 & 287 & 10/2001 & $532$\\
      12/2002 & 287 & 10/2001 & $125$\\
      02/2003 & 937 & 03/2001 & $57$\\
      01/2004 & 287 & 10/2001 & $18$\\
      05/2002 & 456 & 03/2002 & $717$\\
      08/2003 & 456 & 03/2002 & $13$\\
      04/2004 & 456 & 03/2002 & $72$\\
      07/2003 & 101 & 07/2003 & $440$\\
      01/2004 & 867 & 03/2003 & $120$\\
      04/2004 & 200 & 02/2004 & $400$\\
      10/2004 & 956 & 08/2004 & $220$\\
      \bottomrule
    \end{tabular}
  \end{table}
  \begin{enumerate}
  \item Construire les triangles des paiements cumulés.
  \item Quel est le nombre de périodes nécessaires pour observer le
    développement complet des paiements?
  \end{enumerate}
  \begin{rep}
    \begin{enumerate}
      \stepcounter{enumii}
    \item $5$
    \end{enumerate}
  \end{rep}
  \begin{sol}
    \begin{enumerate}
    \item Le triangle des paiements cumulés est présenté dans le
      \autoref{tab:presentation:tri1}. Pour la case $(i, j)$, il suffit de
      sommer les montants des paiements pour les sinistres survenus
      pendant l'année~$i$ et payés pendant la $j\ieme$~période après
      la survenance.
      \begin{table}[!h]
        \centering
        \caption{Paiements réalisés par l'assureur YTR}
        \label{tab:presentation:tri1}
        \begin{tabular}{cccccc}
          \toprule
          & $1$ & $2$ & $3$ & $4$ & $5$\\
          \midrule
          2000 & $425$ & $522$ & $612$ & $714$ & $730$\\
          2001 & $532$ & $657$ & $714$ & $732$ & -\\
          2002 & $717$ & $730$ & $802$ & - & -\\
          2003 & $440$ & $560$ & - & - & -\\
          2004 & $620$ & - & - & - & -\\
          \bottomrule
        \end{tabular}
      \end{table}
    \item Selon les données enregistrées par la compagnie, il est
      raisonnable de supposer que les paiements sont pratiquement
      complets après $5$~périodes de développement. Idéalement, le
      fait de pouvoir consulter une base de données semblables plus
      mature permettrait de confirmer ou d'infirmer cette hypothèse.
      Enfin, connaître le type de portefeuille (assurance automobile -
      dommage matériel, assurance automobile - dommage corporel,
      assurance responsabilité professionnelle, etc.) permettrait
      également d'avoir une idée du nombre de périodes de
      développement nécessaires.
    \end{enumerate}
  \end{sol}
\end{exercice}

\begin{exercice}
  On considère les primes acquises et les montants totaux payés
  suivants, à la fin de 2005, pour les années 2000 à 2005.
  \begin{center}
    \begin{tabular}{|c | c c c |}\hline
      Année & Primes acquises & Montants payés & Rapport sinistres/primes espéré\\ \hline
      2000 & $\numprint{200000}$ & $\numprint{158000}$ & $85\%$\\
      2001 & $\numprint{205000}$ & $\numprint{150000}$ & $87,5\%$\\
      2002 & $\numprint{210000}$ & $\numprint{145000}$ & $85\%$\\
      2003 & $\numprint{212500}$ & $\numprint{140000}$ & $78\%$\\
      2004 & $\numprint{220000}$ & $\numprint{125000}$ & $80\%$\\
      2005 & $\numprint{215000}$ & $\numprint{112000}$ & $75\%$ \\ \hline
    \end{tabular}
  \end{center}
  En utilisant la méthode des rapports sinistres/primes espérés,
  estimer la réserve totale pour les années 2000 à 2005.
  \begin{rep}
    $\numprint{200875}$
  \end{rep}
  \begin{sol}
    On a
    \begin{align*}
      \text{Pertes ultimes estimées}_{2000} &= \text{Rapport
                                              sinistres/primes espéré}_{2000} \times \text{Primes
                                              acquises}_{2000}\\
                                            &= (0,85)(\numprint{200000}) = \numprint{170000}\\
      \text{Pertes ultimes estimées}_{2001} &= \numprint{179375}\\
      \text{Pertes ultimes estimées}_{2002} &= \numprint{178500}\\
      \text{Pertes ultimes estimées}_{2003} &= \numprint{165750}\\
      \text{Pertes ultimes estimées}_{2004} &= \numprint{176000}\\
      \text{Pertes ultimes estimées}_{2005} &= \numprint{161250}\\
      \text{Pertes ultimes estimées}_{Total} &= \numprint{1030875}\\
      \text{Réserve estimée}_{total} &= \text{Pertes ultimes estimées}_{total} - \text{montants payés}_{total}\\
                                            &=  \numprint{1030875} - \numprint{830000} = \numprint{200875}.
    \end{align*}
  \end{sol}
\end{exercice}

\begin{exercice}
  On considère les primes acquises et les réserves individuelles de
  chaque sinistre, à la fin de 2005, pour les années 2000 à 2005
  \begin{center}
    \begin{tabular}{|c | c c|}\hline
      Année & Primes acquises & Somme des réserves individuelles \\ \hline
      2000 & $\numprint{100000}$ & $\numprint{58000}$\\
      2001 & $\numprint{105000}$ & $\numprint{50000}$\\
      2002 & $\numprint{110000}$ & $\numprint{45000}$\\
      2003 & $\numprint{112500}$ & $\numprint{40000}$\\
      2004 & $\numprint{120000}$ & $\numprint{25000}$\\
      2005 & $\numprint{115000}$ & $\numprint{12000}$ \\ \hline
    \end{tabular}
  \end{center}
  En utilisant la méthode des réserves enregistrées, estimer la
  réserve totale pour les années 2000 à 2005 si on utilise un facteur
  de $5\%$.
  \begin{rep}
    $\numprint{241500}$
  \end{rep}
  \begin{sol}
    La somme des réserves individuelles est $\numprint{230000}$.
    Ainsi, la réserve totale est
    $(\numprint{230000})(1,05) = \numprint{241500}$.
  \end{sol}
\end{exercice}

\begin{exercice}
  L'actuaire de la compagnie \textbf{GFR} possède les informations
  suivantes sur les montants payés cumulatifs, en fonction des années
  de développement et de l'année de survenance du sinistre:
  \begin{center}
    \begin{tabular}{|l|l l l l l l l|}\hline
      Année & $1$ & $2$ & $3$ & $4$ & $5$ & $6$ & $7$\\ \hline
      1993 & $\numprint{1780}$ & $\numprint{2673}$ & $\numprint{2874}$ & $\numprint{3094}$ & $\numprint{3157}$ & $\numprint{3166}$ & $\numprint{3166}$ \\
      1994 & $\numprint{3226}$ & $\numprint{4219}$ & $\numprint{4532}$ & $\numprint{4881}$ & $\numprint{5144}$ & $\numprint{5199}$ & \\
      1995 & $\numprint{3652}$ & $\numprint{4989}$ & $\numprint{5762}$ & $\numprint{6436}$ & $\numprint{6720}$ & & \\
      1996 & $\numprint{2723}$ & $\numprint{4301}$ & $\numprint{5526}$ & $\numprint{6231}$ & & & \\
      1997 & $\numprint{2923}$ & $\numprint{4666}$ & $\numprint{5349}$ & & & & \\
      1998 & $\numprint{2990}$ & $\numprint{5417}$ & & & & & \\
      1999 & $\numprint{3917}$ & & & & & &\\ \hline
    \end{tabular}
  \end{center}

  \begin{enumerate}
  \item Combien de dollars ont été payés pour les sinistres survenus
    en 1993 ?
  \item En 1997, combien de dollars ont été payés pour les sinistres
    survenus en 1995 ?
  \item Quel est le montant total de sinistres payés pour l'année de
    survenance 1998 ?
  \item Combien de dollars ont été payés en indemnités pendant l'année
    1999 ?
  \item Combien de dollars ont été payés en indemnités pendant l'année
    1994 ?
  \end{enumerate}

  \begin{rep}
    \begin{enumerate}
    \item $\numprint{3166}$
    \item $\numprint{773}$
    \item $\numprint{5417}$
    \item $\numprint{8071}$
    \item $\numprint{4119}$
    \end{enumerate}
  \end{rep}
  \begin{sol}
    \begin{enumerate}
    \item Le dernier montant cumulatif disponible pour l'année de
      survenance 1993 est $\numprint{3166}\$$.
    \item On calcule $\numprint{5762} - \numprint{4989} = 773$.
    \item Le dernier montant cumulatif disponible pour l'année de
      survenance 1998 est $\numprint{5417}$.
    \item On calcule $\numprint{3917} +
      (\numprint{5417}-\numprint{2990}) + (\numprint{5349} -
      \numprint{4666}) + (\numprint{6231} - \numprint{5526}) +
      (\numprint{6720}-\numprint{6436}) + (\numprint{5199} -
      \numprint{5144}) + 0 = \numprint{8071}$.
    \item On calcule $\numprint{3226} +
      (\numprint{2673}-\numprint{1780}) = \numprint{4119}$.
    \end{enumerate}
  \end{sol}
\end{exercice}

\Closesolutionfile{solutions}
\Closesolutionfile{reponses}

%%% Local Variables:
%%% mode: latex
%%% TeX-master: "provisionnement-assurance-iard"
%%% TeX-engine: xetex
%%% coding: utf-8
%%% End:

%%% Copyright (C) 2020 Vincent Goulet, Frédérick Guillot, Mathieu Pigeon
%%%
%%% Ce fichier fait partie du projet
%%% «Provisionnement en assurance IARD»
%%% https://gitlab.com/vigou3/provisionnement-assurance-iard
%%%
%%% Cette création est mise à disposition sous licence
%%% Attribution-Partage dans les mêmes conditions 4.0
%%% International de Creative Commons.
%%% https://creativecommons.org/licenses/by-sa/4.0/

\chapter{Modélisation déterministe des provisions}
\label{chap:deterministe}

\section{Modèle Chain-Ladder}
\label{sec:deterministe:CL}

Ce premier modèle et ses variantes sont considérés comme des méthodes
\textbf{déterministes}, c'est-à-dire qu'elles n'utilisent pas de
distribution ou de propriété statistique. Ils reposent sur l'hypothèse
de stabilité du délai s'écoulant entre la survenance d'un sinistre et
le règlement. Ainsi, sont exclus de la modélisation
\begin{itemize}
\item les effets de l'inflation;
\item les changements de structure du portefeuille;
\item les changements des contrats d'assurance;
\item les changement dans la gestion des sinistres.
\end{itemize}

Le modèle Chain-Ladder est simple et intuitif, ne considère que le
triangle des coûts encourus et se base sur l'observation de
l'évolution de l'encouru cumulatif d'une période de développement à
l'autre: si l'encouru cumulatif augmente d'un certain pourcentage
d'une période à la période suivante, on suppose que le même phénomène
devrait se reproduire pour les années d'accident futures.

Les paramètres $\lambda_j$ représentant les pourcentages
d'augmentation sont appelés \emph{facteurs multiplicatifs}, ou bien
\emph{facteurs de déroulement}, ou encore \emph{facteurs de
  développement}. L'indice $j$ représente le passage de la période $j$
à la période $j + 1$, ou encore de la colonne $j$ à la colonne $j+1$
dans un triangle de développement. Le modèle sous-jacent est donc que:
$C_{i, j+1} = \lambda_j C_{i, j}$. Les coefficients $\lambda_j$,
$j = 1, \dots, J - 1$, sont estimés à partir des observations par une
moyenne pondérée des facteurs de développements par année d'accident:
\begin{align*}
  \hat{\lambda}_j
  &= \frac{\sum_{i = 1}^{I - j} C_{i, j + 1}}{%
    \sum_{i = 1}^{I - j} C_{i, j}} \\
  &= \sum_{i = 1}^{I - j} \frac{C_{i, j}}{C_{\pt, j}}
    \frac{C_{i, j + 1}}{C_{i, j}}, \quad
    C_{\pt, j} = \sum_{k = 1}^{I - j} C_{k, j}.
\end{align*}
L'estimateur Chain-Ladder de $C_{i, j}$ pour $i + j - 1 > I$ (partie
inférieure du triangle) est alors:
\begin{equation*}
  \hat{C}_{i, j}^{\text{CL}}
  = C_{i, I - j + 1} \hat{\lambda}_{I - j + i} \cdots \hat{\lambda}_{j - 1}.
\end{equation*}

Les provisions correspondent aux montants encore à payer pour les
sinistres. La provision Chain-Ladder pour l'année d'accident $i$ est
égale à la différence entre les sinistres ultimes et les sinistres
payés en date d'évaluation:
\begin{equation*}
  \hat{R}_i^{\text{CL}} = \hat{C}_{i, J}^{\text{CL}} - C_{i, I - i + 1}.
\end{equation*}
La provision totale, quant à elle, est simplement la somme des
provisions par année d'accident:
\begin{align*}
  \hat{R}^{\text{CL}}
  &= \sum_{i = 1}^I \hat{R}_i^{\text{CL}} \\
  &= \sum_{i = 1}^I (\hat{C}_{i, J}^{\text{CL}} - C_{i, I - i + 1}).
\end{align*}

\begin{exemple}
  \label{ex:deterministe:CL1}
  Le \autoref{tab:deterministe:CL1:donnees} contient les données d'un
  triangle de développement pour cinq années d'accident et autant
  d'années de développement.

  \begin{table}
    \centering
    \caption{Triangle de développement pour
      l'\autoref{ex:deterministe:CL1}}
    \label{tab:deterministe:CL1:donnees}
    \begin{tabular}{crrrrr}
      \toprule
      & \multicolumn{5}{c}{Développement (âge)} \\
      Année & 1 & 2 & 3 & 4 & 5 \\
      \midrule
      1 & 100 & 150 & 175 & 180 & 200 \\
      2 & 110 & 168 & 192 & 205 \\
      3 & 115 & 169 & 202 \\
      4 & 125 & 185 \\
      5 & 150 \\
      \bottomrule
    \end{tabular}
  \end{table}

  Les estimateurs de la méthode Chain-Ladder des facteurs de
  développement sont:
  \begin{align*}
    \hat{\lambda}_4 &= \frac{200}{180} = 1,111 \\
    \hat{\lambda}_3 &= \frac{180+205}{175+192} = 1,049 \\
    \hat{\lambda}_2 &= \frac{175+192+202}{150+168+169} = 1,168 \\
    \hat{\lambda}_1 &= \frac{150+168+169+185}{100+110+115+125} = 1,493.
  \end{align*}
  À partir de ces facteurs de développement, nous pouvons calculer les
  prévisions des montants cumulatifs sous la diagonale du triangle de
  développement. Voici quelques exemples de calcul:
  \begin{align*}
    \hat{C}_{2,5}^{\text{CL}}
    &= C_{2, 4} \hat{\lambda}_4 &
    \hat{C}_{3,4}^{\text{CL}}
    &= C_{3, 3} \hat{\lambda}_3 \\
    &= 205 (1,111) &
    &= 202 (1,049) \\
    &= 227,78 &
    &= 211,91 \\
    \hat{C}_{3,5}^{\text{CL}}
    &= \hat{C}_{3, 4}^{\text{CL}} \hat{\lambda}_4 = C_{3, 3} \hat{\lambda}_3 \hat{\lambda}_4 &
    \hat{C}_{4,5}^{\text{CL}}
    &= C_{4, 2} \hat{\lambda}_2 \hat{\lambda}_3 \hat{\lambda}_4 \\
    &= 202 (1,049)(1,111) &
    &= 185 (1,169)(1,049)(1,111) \\
    &= 235,45 &
    &= 251,95.
  \end{align*}
  Les résultats complets se trouvent dans le
  \autoref{tab:deterministe:CL1:complet}.

  \begin{table}
    \centering
    \caption{Triangle de développement complété pour
      l'\autoref{ex:deterministe:CL1}}
    \label{tab:deterministe:CL1:complet}
    \begin{tabular}{crrrrr}
      \toprule
      & \multicolumn{5}{c}{Développement (âge)} \\
      Année & 1 & 2 & 3 & 4 & 5 \\
      \midrule
      1 & 100 & 150    & 175    & 180    & 200    \\
      2 & 110 & 168    & 192    & 205    & 227,78 \\
      3 & 115 & 169    & 202    & 211,91 & 235,45 \\
      4 & 125 & 185    & 216,15 & 226,75 & 251,95 \\
      5 & 150 & 224,00 & 261,72 & 274,55 & 305,06 \\
      \bottomrule
    \end{tabular}
  \end{table}

  Nous pouvons maintenant calculer les provisions par année d'accident
  ainsi que la provision totale:
  \begin{align*}
    \hat{R}_1^{\text{CL}}
    &= C_{1, 5} - C_{1, 5} = 0 \\
    \hat{R}_2^{\text{CL}}
    &= \hat{C}_{2, 5}^{\text{CL}} - C_{2, 4} = 22,78 \\
    \hat{R}_3^{\text{CL}}
    &= \hat{C}_{3, 5}^{\text{CL}} - C_{3, 3} = 33,45 \\
    \hat{R}_4^{\text{CL}}
    &= \hat{C}_{4, 5}^{\text{CL}} - C_{4, 2} = 66,95 \\
    \hat{R}_5^{\text{CL}}
    &= \hat{C}_{5, 5}^{\text{CL}} - C_{5, 1} = 155,06 \\
    \intertext{et}
    \hat{R}^{\text{CL}}
    &= 22,78 + 33,45 + 66,95 + 155,06 = 278,24.
  \end{align*}
  Nous pouvons regrouper l'ensemble des résultats ci-dessus comme dans
  le \autoref{tab:deterministe:CL1:resultats}. %
  \qed

  \begin{table}
    \centering
    \caption{Tableau sommaire des résultats pour
      l'\autoref{ex:deterministe:CL1}}
    \label{tab:deterministe:CL1:resultats}
    \begin{tabular}{crrrrrr}
      \toprule
      & \multicolumn{5}{c}{Développement (âge)} \\
      Année & 1 & 2 & 3 & 4 & 5 & Provision \\
      \midrule
      1 & 100 & 150    & 175    & 180    & 200    &   0,00 \\
      2 & 110 & 168    & 192    & 205    & 227,78 &  22,78 \\
      3 & 115 & 169    & 202    & 211,91 & 235,45 &  33,45 \\
      4 & 125 & 185    & 216,15 & 226,75 & 251,95 &  66,95 \\
      5 & 150 & 224,00 & 261,72 & 274,55 & 305,06 & 155,06 \\
      \midrule
      $\hat{\lambda}_j$ & $1,493$ & $1,168$ & $1,049$ & $1,111$ \\
      \midrule
      TOTAL & & & & & & 278,24 \\
      \bottomrule
    \end{tabular}
  \end{table}
\end{exemple}

\begin{exemple}
  \label{ex:deterministe:CL2}
  Le \autoref{tab:deterministe:CL2:donnees} contient les données d'un
  triangle de paiements pour les années d'accident 1997--2001. Nous
  devons calculer la provision totale de l'assureur à partir de ces
  données.

  \begin{table}
    \centering
    \caption{Triangle de paiements pour
      l'\autoref{ex:deterministe:CL2}}
    \label{tab:deterministe:CL2:donnees}
    \begin{tabular}{crrrrr}
      \toprule
      & \multicolumn{5}{c}{Paiments} \\
      Année & $1997$ & $1998$ & $1999$ & $2000$ & $2001$  \\
      \midrule
      $1997$ & $\nombre{26312}$ & $\nombre{31467}$ & $\nombre{24672}$ & $\nombre{13055}$ & $\nombre{6158}$ \\
      $1998$ & $\nombre{30470}$ & $\nombre{35012}$ & $\nombre{25491}$ & $\nombre{12589}$ \\
      $1999$ & $\nombre{49756}$ & $\nombre{51831}$ & $\nombre{35267}$ \\
      $2000$ & $\nombre{50420}$ & $\nombre{52315}$ \\
      $2001$ & $\nombre{56762}$ \\
      \bottomrule
    \end{tabular}
  \end{table}

  En premier lieu, il faut transformer le triangle des paiements en
  triangle cumulatif, comme l'exige la méthode Chain-Ladder. Le
  triangle cumulatif se trouve au
  \autoref{tab:deterministe:CL2:donnees:cumulatives}.

  \begin{table}
    \centering
    \caption{Triangle de paiements cumulatifs pour
      l'\autoref{ex:deterministe:CL2}}
    \label{tab:deterministe:CL2:donnees:cumulatives}
    \begin{tabular}{crrrrr}
      \toprule
      & \multicolumn{5}{c}{Paiments cumulatifs (développement)} \\
      Année & $1997$ & $1998$ & $1999$ & $2000$ & $2001$  \\
      \midrule
      $1997$ & $\nombre{26312}$ & $\nombre{57779}$  & $\nombre{82451}$ & $\nombre{95506}$ & $\nombre{101664}$\\
      $1998$ & $\nombre{30470}$ & $\nombre{65482}$  & $\nombre{90973}$ & $\nombre{103562}$ \\
      $1999$ & $\nombre{49756}$ & $\nombre{101587}$ & $\nombre{136854}$	\\
      $2000$ & $\nombre{50420}$ & $\nombre{102735}$ \\
      $2001$ & $\nombre{56762}$ \\
      \bottomrule
    \end{tabular}
  \end{table}

  Afin de simplifier la notation des indices, nous allons supposer que
  1997 est l'année $1$. Nous avons donc $I = J = 5$. Voici un
  exemple de calcul de facteur de déroulement:
  \begin{align*}
    \hat{\lambda}_2
    &= \frac{\sum_{i = 1}^3 C_{i, 3}}{\sum_{i = 1}^3 C_{i, 2}}\\
    &= \frac{C_{1, 3}+ C_{2, 3}+ C_{3, 3}}{C_{1, 2}+ C_{2, 2}+ C_{3, 2}} \\
    &= \frac{\nombre{82451} + \nombre{90973} +
      \nombre{136854}}{\nombre{57779} + \nombre{65482} +
      \nombre{101587}} \\
    &= 1,380.
  \end{align*}
  Ce résultat signifie que pour une année de survenance donnée, le
  total des règlements au bout de trois ans doit être $38~\%$
  supérieur à celui au bout de deux ans.

  Le \autoref{tab:deterministe:CL2:facteurs} regroupe les facteurs de
  déroulement des quatre années de développement ainsi que les valeurs
  qui servent à les calculer. À partir de ces résultats, nous pouvons
  compléter la partie inférieure du triangle de développement; les
  résultats se trouvent au \autoref{tab:deterministe:CL2:complet}. Le
  montant total des provisions pour l'année 1999 est donc
  $\nombre{167119} - \nombre{136854} = \nombre{30265}$~\$, alors que
  la provision totale s'élève à $\nombre{250520}$~\$. %
  \qed

  \begin{table}
    \centering
    \caption{Données pour le calcul des facteurs de déroulement dans
      l'\autoref{ex:deterministe:CL2}}
    \label{tab:deterministe:CL2:facteurs}
    \begin{tabular}{lrrrr}
      \toprule
      & \multicolumn{4}{c}{$j$} \\
      Quantité & $1$ & $2$ & $3$ & $4$ \\
      \midrule
      $\sum_{i=1}^{I - j} C_{i, j+1}$ & $\nombre{327583}$ & $\nombre{310278}$ & $\nombre{199068}$ & $\nombre{101664}$ \\
      $\sum_{i = 1}^{I - j} C_{i, j}$ & $\nombre{156958}$ & $\nombre{224848}$ & $\nombre{173424}$ & $\nombre{95506}$ \\
      $\hat{\lambda_j}$ & $2,087$ & $1,380$ & $1,148$ & $1,064$ \\
      \bottomrule
    \end{tabular}
  \end{table}

  \begin{table}
    \centering
    \caption{Triangle de de développement complété pour
      l'\autoref{ex:deterministe:CL2}}
    \label{tab:deterministe:CL2:complet}
    \begin{tabular}{crrrrr}
      \toprule
      & \multicolumn{5}{c}{Paiments cumulatifs (développement)} \\
      Année & $1997$ & $1998$ & $1999$ & $2000$ & $2001$  \\
      \midrule
      Année & $1$ & $2$ & $3$ & $4$ & $5$  \\ \hline
      $1997$ & $\nombre{26312}$ & $\nombre{57779}$  & $\nombre{82451}$  & $\nombre{95506}$  & $\nombre{101664}$\\
      $1998$ & $\nombre{30470}$ & $\nombre{65482}$  & $\nombre{90973}$  & $\nombre{103562}$ & $\nombre{110239}$\\
      $1999$ & $\nombre{49756}$ & $\nombre{101587}$ & $\nombre{136854}$ & $\nombre{157090}$ & $\nombre{167219}$\\
      $2000$ & $\nombre{50420}$ & $\nombre{102735}$ & $\nombre{141769}$ & $\nombre{162732}$ & $\nombre{173224}$\\
      $2001$ & $\nombre{56762}$ & $\nombre{118467}$ & $\nombre{163477}$ & $\nombre{187651}$ & $\nombre{199750}$ \\
      \bottomrule
    \end{tabular}
  \end{table}
\end{exemple}

\subsection{Mise à jour des estimations}
\label{sec:deterministe:CL:maj}

Jusqu'à maintenant, nous avons effectué le calcul des provisions
uniquement à la fin de la dernière année d'accident. En pratique,
l'exercice sera répété à chaque bilan comptable, c'est-à-dire à la fin
de chaque année d'accident. Cela permet d'effectuer annuellement une
mise à jour des estimations.

\begin{exemple}
  \label{ex:deterministe:CL3}
  Reprenons le contexte de l'~\autoref{ex:deterministe:CL2} où la
  provision totale calculée par la méthode Chain-Ladder s'élève à
  $\nombre{250520}$~\$. À la fin de l'année 2002, une nouvelle
  diagonale est ajoutée au triangle de développement du
  \autoref{tab:deterministe:CL2:donnees:cumulatives}. Le nouveau
  triangle de développement se trouve au
  \autoref{tab:deterministe:CL3:donnees}.

  \begin{table}
    \centering
    \caption{Triangle de paiements cumulatifs pour
      l'\autoref{ex:deterministe:CL3} suite à l'ajout d'une année
      d'accident}
    \label{tab:deterministe:CL3:donnees}
    \begin{tabular}{crrrrrr}
      \toprule
      & \multicolumn{6}{c}{Développement (âge)} \\
      Année & $1997$ & $1998$ & $1999$ & $2000$ & $2001$ & $2002$ \\
      \midrule
      $1997$ & $\nombre{26312}$ & $\nombre{57779}$  & $\nombre{82451}$ & $\nombre{95506}$ & $\nombre{101664}$ & $\nombre{101664}$ \\
      $1998$ & $\nombre{30470}$ & $\nombre{65482}$  & $\nombre{90973}$ & $\nombre{103562}$ & $\nombre{113455}$ \\
      $1999$ & $\nombre{49756}$ & $\nombre{101587}$ & $\nombre{136854}$	& $\nombre{160233}$ \\
      $2000$ & $\nombre{50420}$ & $\nombre{102735}$ & $\nombre{138653}$ \\
      $2001$ & $\nombre{56762}$ & $\nombre{123156}$ \\
      $2002$ & $\nombre{61262}$ \\
      \bottomrule
    \end{tabular}
  \end{table}

  Le \autoref{tab:deterministe:CL3:paiements} compare les paiements
  effectués dans l'année de calendrier 2002 à ceux prédits par la
  méthode Chain-Ladder. Nous avons donc sous-estimé les paiements par
  $\nombre{135584} - \nombre{127652} = \nombre{7932}~\$$.

  \begin{table}
    \centering
    \caption{Paiements prédits pour 2002 par la méthode Chain-Ladder
      pour les années d'accident 1997--2001 pour les données de
      l'\autoref{ex:deterministe:CL3} et paiements réellement
      effectués}
    \label{tab:deterministe:CL3:paiements}
    \begin{tabular}{crr}
      \toprule
      Année & Prédit & Réalisé \\
      \midrule
      1998 & $\nombre{ 6677}$ & $\nombre{ 9893}$ \\
      1999 & $\nombre{20236}$ & $\nombre{23379}$ \\
      2000 & $\nombre{39034}$ & $\nombre{35918}$ \\
      2001 & $\nombre{61705}$ & $\nombre{66394}$ \\
      \midrule
      Total & $\nombre{127652}$ & $\nombre{135584}$ \\
      \bottomrule
    \end{tabular}
  \end{table}

  Nous pouvons aussi calculer les provisions par année d'accident avec
  le nouveau triangle de développement et les comparer avec celles
  calculées précédemment. La \autoref{tab:deterministe:CL3:provisions}
  dresse ce bilan. Le montant total de provisions nécessaires pour les
  années antérieures nécessite un apport de $\nombre{21884}$~\$, ce
  qui implique une correction de ce montant sur les exercices
  financiers antérieurs. %
  \qed

  \begin{table}
    \centering
    \caption{Provisions des années 1997--2001 calculées au terme de
      l'exercice 2001 et au terme de l'exercice 2002 par la méthode
      Chain-Ladder pour les données de
      l'\autoref{ex:deterministe:CL3}}
    \label{tab:deterministe:CL3:provisions}
    \begin{tabular}{crrr}
      \toprule
      Année & Exercice 2001 & Exercice 2002 & Écart \\
      \midrule
      1997 & $\nombre{101664}$ & $\nombre{101664}$ & $0$\\
      1998 & $\nombre{110239}$ & $\nombre{113455}$ & $-\nombre{3216}$ \\
      1999 & $\nombre{167219}$ & $\nombre{173153}$ & $-\nombre{5934}$ \\
      2000 & $\nombre{173224}$ & $\nombre{173506}$ & $-\nombre{1282}$ \\
      2001 & $\nombre{199750}$ & $\nombre{211202}$ & $-\nombre{11452}$ \\
      \midrule
      Total & & & $-\nombre{21884}$ \\
      \bottomrule
    \end{tabular}
  \end{table}
\end{exemple}

\subsection{Inconvénients de la méthode Chain-Ladder}
\label{sec:deterministe:CL:inconvenients}

Malgré le caractère intuitif et simple de la méthode, celle-ci
comporte plusieurs défauts importants. En voici quelques uns.
\begin{itemize}
\item Le développement est identique pour toutes les années de
  survenance, ce qui n'est pas le cas en pratique s'il y a des
  changements dans la jurisprudence et/ou dans le management de la
  compagnie.
\item Pour les années récentes, l'incertitude est très importante car
  l'évaluation de la provision correspondra au produit de plusieurs
  estimateurs.
\item Il est impossible d'effectuer une évaluation de la précision de
  l'estimation puisque la méthode est déterministe. Il pourrait être
  intéressant (et plus que souhaitable) que les méthodes utilisées
  indiquent des intervalles de confiance pour les provisions.
\end{itemize}

\subsection{Variantes}
\label{sec:deterministe:CL:variantes}

Afin de solutionner certains problèmes connus de la méthode
Chain-Ladder pure, il est fréquent que les praticiens utilisent une
pondération des données lors de l'estimation des facteurs de
déroulement. Quelques exemples de pondérations typiquement utilisées
en pratique:
\begin{itemize}
\item plus de poids aux années récentes et moins aux années éloignées;
\item pondération en relation avec l'exposition au risque;
\item choix des données en excluant les valeurs extrêmes;
\item utilisation d'autres types de moyennes (géométrique, harmonique,
  etc.), chacune ayant ses avantages et ses inconvénients;
\item calcul de diverses statistiques pour évaluer les changements
  dans la jurisprudence et dans les méthodes de management.
\end{itemize}
Dans un contexte pratique, un actuaire utilisera souvent diverses
pondérations et choisira des facteurs de déroulement selon son
jugement.

Il est important de noter que lorsque les facteurs de déroulement
$\lambda_j$ ne sont plus estimés par comme les moyennes pondérées
\begin{equation*}
  \hat{\lambda}_j =
  \frac{\sum_{i = 1}^{I-j} C_{i, j + 1}}{%
    \sum_{i = 1}^{I - j} C_{i,k}}, \quad j = 1, \dots, I - 1,
\end{equation*}
alors on ne parle plus de la technique Chain-Ladder, mais bien de
l'une de ses variantes. Formellement, la méthode appelée Chain-Ladder
ne correspond qu'à l'équation ci-dessus.

\begin{exemple}
  \label{ex:deterministe:variantes}
  À partir du \autoref{tab:deterministe:CL2:donnees:cumulatives} de
  l'\autoref{ex:deterministe:CL2}, nous pouvons calculer les facteurs
  de déroulement $\hat{\lambda}_{i, j}$ pour passer, pour l'année
  d'accident $i = 1, \dots, I$, de l'âge de développement $j$ à l'âge
  $j + 1$, $j = 1, \dots, I - i$. Le
  \autoref{tab:deterministe:variantes} montre divers choix possibles
  pour calculer, à partir de ces statistiques, des facteurs de
  déroulement $\hat{\lambda}_j$ uniques par âge de développement. La
  moyenne pondérée correspond à la méthode Chain-Ladder.

  \begin{table}
    \centering
    \caption{Facteurs de déroulement $\hat{\lambda}_{i, j}$ pour les
      données du \autoref{tab:deterministe:CL2:donnees:cumulatives}
      (haut) et divers choix possibles de facteurs $\hat{\lambda}_j$
      (bas)}
    \label{tab:deterministe:variantes}
    \begin{tabular}{lrrrr}
      \toprule
      Année & 1 & 2 & 3 & 4 \\
      \midrule
      1997 &$2,196$ & $1,427$ & $1,158$ & $1,064$\\
      1998 &$2,149$ & $1,389$ & $1,138$ \\
      1999 &$2,042$ & $1,347$ \\
      2000 &$2,038$ \\
      \midrule
      Moyenne arithmétique & $2,106$ & $1,388$ & $1,148$ & $1,064$ \\
      Moyenne géométrique  & $2,104$ & $1,387$ & $1,148$ & $1,064$ \\
      Moyenne pondérée     & $2,087$ & $1,380$ & $1,148$ & $1,064$ \\
      Minimum              & $2,038$ & $1,347$ & $1,138$ & $1,064$ \\
      Maximum              & $2,196$ & $1,427$ & $1,158$ & $1,064$ \\
      \bottomrule
    \end{tabular}
  \end{table}

  La tendance actuelle de l'industrie privée consiste à laisser
  l'actuaire choisir d'instinct les facteurs de déroulement
  $\lambda_j$. %
  \qed
\end{exemple}


\section{Bornhuetter-Ferguson}
\label{sec:deterministe:BF}

La méthode de provisionnement de Bornhuetter-Ferguson est basée sur le
célèbre article de \citet{Bornhuetter}. Le but de cette méthode
est d'assurer une meilleure stabilité de l'estimation des provisions
pour les jeunes années de survenance, celles qui dépendent beaucoup des
premiers paiements. La méthode proposée se décompose en trois étapes:
\begin{enumerate}
\item déterminer les pertes ultimes espérées;
\item calculer les taux de déroulement en utilisant la méthode
  Chain-Ladder;
\item déterminer les proportions du montant ultime qu'il reste à
  payer.
\end{enumerate}

Les pertes ultimes espérées peuvent être calculées selon le nombre de
polices vendues, les primes reçues, ou bien établies directement à
partir du jugement de l'actuaire. Dans tous les cas, l'information
peut provenir d'une source extérieure aux données du triangle. De
manière historique, les actuaires utilisent surtout la \textit{méthode
  des rapports sinistres/primes espérés} pour trouver les pertes
ultimes.

La méthode Bornhuetter-Ferguson utilise une procédure analogue à celle
de la méthode Chain-Ladder pour le calcul des taux de déroulement. Ces
taux de déroulement sont utilisés avec les pertes ultimes pour
extrapoler les pertes dans le futur. Autrement dit, la structure des
facteurs de déroulement est utilisée pour exprimer l'encouru cumulatif
comme un pourcentage des pertes ultimes. On rappelle que pour la
méthode Chain-Ladder, on avait
\begin{equation*}
  \hat{C}_{i, J}^{\text{CL}} = C_{i, j} \prod_{k = j}^{I-1} \lambda_k,
\end{equation*}
où $\prod_{k=j}^{I-1} \lambda_k $ représente le produit des facteurs
de déroulement jusqu'à la période ultime, $C_{i, I}$ est l'encouru
ultime et $C_{i, j}$ est l'encouru cumulatif de l'année de survenance
$i$ à la période de déroulement $j$. La provision $R_i$ de l'année
d'accident $i$ à partir de la période de développement $j$ de la
méthode Chain-Ladder est, par définition, la différence entre
l'encouru ultime et l'encouru à la période de déroulement $j$:
\begin{equation*}
  R_i^{\text{CL}} = \hat{C}_{i, I}^{\text{CL}} - C_{i, j}.
\end{equation*}
En combinant ces deux dernières équations, on obtient
\begin{equation*}
  R_i^{\text{CL}} = \hat{C}_{i, I}
  \left(
    1 - \frac{1}{\prod_{k=j}^{I-1} \lambda_k}
  \right) =  \hat{C}_{i, I}^{\text{CL}} (1 - \beta_j),
\end{equation*}
où $\beta_j = 1/\prod_{k=j}^{I-1} \lambda_k$ a été utilisé afin
de simplifier la notation. La variable $\hat{C}_{i,n}$ correspond à la
perte ultime espérée de l'année de survenance $i$. Alors que la
méthode Chain-Ladder utilise les données du triangle et les
$\lambda_k$ pour trouver la valeur de $\hat{C}_{i,I}$, la méthode
Borhuetter-Ferguson se permet d'utiliser une source extérieure aux
données du triangle pour son estimation.

Pour trouver les pertes incrémentales $Y_{i,j}$ prévues pour une année
d'accident et une période donnée, on utilise la même logique. D'abord,
on trouve les provisions à partir du début de la période jusqu'à la
période ultime. Ensuite, on trouve les provisions à partir de la fin de
la période ciblée jusqu'à la période ultime. La différence entre les
deux provisions représente les pertes incrémentales prévues, notées
$Y_{i,j}$.

\begin{exemple}
  On suppose que les facteurs de développement suivants ont été
  estimés par la méthode Chain-Ladder:
  \begin{center}
    \begin{tabular}{|c c c c c|}\hline
      1$\rightarrow$ 2 &  2$\rightarrow$ 3 &  3$\rightarrow$ 4 &  4$\rightarrow$ 5  &5 $\rightarrow  \infty$  \\ \hline
      $1,41$ & $1,22$ & $1,16$ & $1,08$ & $1,04$ \\ \hline
    \end{tabular}
  \end{center}
  et que
  \begin{itemize}
  \item pour l'année 1998, on estime le 31/12/1999 des paiements
    cumulatifs de $\nombre{420 000}$~\$;
  \item les primes acquises de 1998 sont de $\nombre{1 000 000}$~\$;
    et
  \item le rapport sinistres/primes espéré est de $0,600$.
  \end{itemize}
  Estimer les provisions pour l'année d'accident 1998
  \begin{enumerate}
  \item en utilisant la méthode du rapport sinistres/primes espéré;
  \item en utilisant la méthode Chain-Ladder; et
  \item en utilisant la méthode Bornhuetter-Ferguson.
  \end{enumerate}

  La première évaluation des provisions de l'année d'accident 1998 est
  le 31 décembre 1998. Ainsi, le 31 décembre 1999, on est à la
  deuxième évaluation.
  \begin{enumerate}
  \item Méthode du rapport sinistres/primes espéré:
    \begin{align*}
      \underbrace{\hat{C}_{i}^{LR}}_{\text{Pertes ultimes attendues selon LR}} &=
                                                                                 \underbrace{\Esp{LR}}_{\text{Rapport sinistres/primes espéré}} \times \text{Primes acquises}_i\\
                                                                               &= 0,60 * \nombre{1000000} = \nombre{600000} \\
      \underbrace{R_i^{LR}}_{\text{Provisions selon la méthode du rapport sinistres/primes}}
                                                                               &= \hat{C}_{i}^{LR} - \underbrace{C_{i, 2}}_{\text{Paiements cumulatifs à la deuxième évaluation}} \\
                                                                               &= \nombre{600000} - \nombre{420000} = \nombre{180000}.
    \end{align*}
    %
  \item Méthode Chain-Ladder:
    \begin{align*}
      \underbrace{\hat{C}_{i}^{CL}}_{\text{Pertes ultimes attendues selon CL}} &=
                                                                                 C_{i, 2} \times (\prod_{k=2}^{\infty} \lambda_k)\\
                                                                               &= \nombre{420000} \times 1,22 * 1,16 * 1,108 * 1,04 \\
                                                                               &= \nombre{420000} * 1,58955 = \nombre{667612}\\
      \underbrace{R_i^{CL}}_{\text{Provisions selon la méthode CL}}
                                                                               &= \hat{C}_{i}^{CL} - C_{i, 2} \\
                                                                               &= \nombre{667612} - \nombre{420000} = \nombre{247612}.
    \end{align*}
    %
  \item Méthode Bornhuetter-Ferguson:
    \begin{align*}
      R_i^{BF} &= \hat{C}_{i}^{LR} \times ( 1 - \frac{1}{\prod_{k=2}^{\infty} \lambda_k}) \\
               &= \nombre{600000} (1-\frac{1}{1,58955}) = \nombre{222535}.
    \end{align*}

    On voit que le $\nombre{222535}$~\$ n'a pas été trouvé à l'aide
    de $C_{i, 2} = \nombre{420000}$~\$.

    Le modèle est très proche de celui de Chain-Ladder. En effet, pour
    la provision totale, Bornhuetter-Ferguson applique le modèle
    Chain-Ladder en supposant un
    $C_{i,2} = \nombre{600000}/1,58955 = \nombre{377465}$ qui
    génèreraient un $C_i^{CL} = \nombre{600000}$ avec les
    $\prod_{k=2}^{\infty} \lambda_k = 1,58955$, au lieu du
    $C_{i, 2} = \nombre{420000}$.
  \end{enumerate}
  \qed
\end{exemple}

\begin{exemple}
  Ventiler les provisions obtenues par année de développement et
  analyser les résultats.

  Pour les trois méthodes:
  \begin{enumerate}
  \item La méthode du rapport sinistres/primes espéré:

    Cette méthode ne permet pas de ventiler les paiements et
    d'analyser son développement.

  \item La méthode Chain-Ladder:

    Les $\lambda_k$ permettent de voir l'évolution des paiements:
    \begin{center}
      \begin{tabular}{|c|c|c|}\hline
        Date ($j$) & $\hat{C}_{i,j}^{CL}$ & $\hat{Y}_{i,j}^{CL}$   \\ \hline
        Décembre 2000 &  $\nombre{420000}*1,22  = \nombre{512400}$ & $\nombre{92400}$ \\ \hline
        Décembre 2001 &  $\nombre{420000}*1,22*1,16 = \nombre{594384}$ & $\nombre{81984}$ \\ \hline
        Décembre 2002 &  $\nombre{420000}*1,22*1,16*1,08  =
                        \nombre{641935}$ & $\nombre{47551}$\\ \hline
        Décembre 2003 &  $\nombre{420000}*1,22*1,16*1,08*1,04 = \nombre{667612}$ & $\nombre{25677}$\\ \hline
      \end{tabular}
    \end{center}

  \item La méthode Bornhuetter-Ferguson:

    Les $\lambda_k$ permettent aussi de voir l'évolution des
    paiements. Pour la provision de l'année d'accident 1998, l'évolution
    des paiements est
    \begin{itemize}
    \item du temps 2 au temps $\infty$, le facteur de développement
      est de $1,22 * 1,16 * 1,08 * 1,04 = 1,58955$;
    \item du temps 3 au temps $\infty$, le facteur de développement
      est de $1,16 * 1,08 * 1,04 = 1,302912$;
    \item du temps 4 au temps $\infty$, le facteur de développement
      est de $1,08 * 1,04 = 1,1232$; et
    \item du temps 5 au temps $\infty$, le facteur de développement
      est de $1,04$.
    \end{itemize}
    Ainsi,
    \begin{center}
      \begin{tabular}{|c|c|c|}\hline
        Date ($j$) & $\hat{C}_{i,j}^{BF}$ & $\hat{Y}_{i,j}^{BF}$  \\ \hline
        Décembre 2000 &  $\nombre{600000} \left((1-\frac{1}{1,58955}) - (1-\frac{1}{1,302912})\right)   + \nombre{420000} = \nombre{503039}$ & $\nombre{83039}$ \\ \hline
        Décembre 2001 &  $\nombre{600000} \left((1-\frac{1}{1,302912}) - (1-\frac{1}{1,1232})\right) + \nombre{503039} = \nombre{576724}$ & $\nombre{73685}$\\ \hline
        Décembre 2002 &  $\nombre{600000} \left((1-\frac{1}{1,1232}) - (1-\frac{1}{1,04})\right) + \nombre{576724} = \nombre{619459}$ & $\nombre{42735}$\\ \hline
        Décembre 2003 &  $\nombre{600000} \left((1-\frac{1}{1,04}) \right)  + \nombre{619459} = \nombre{642535}$ & $\nombre{23077}$\\ \hline
      \end{tabular}
    \end{center}
    Cette méthode permet de stabiliser l'évaluation de la provision,
    mais demande de trouver le rapport sinistres/primes estimé.
  \end{enumerate}
  \qed
\end{exemple}


\section{Méthode London Chain}

Cette méthode se veut une généralisation de la méthode Chain-Ladder.
On se rappelle que la méthode Chain-Ladder suppose la forme suivante:
\begin{equation*}
  C_{i,j+1} = \lambda_j C_{i,j}.
\end{equation*}
Dans le cas de la méthode London Chain, un autre paramètre est ajouté
pour l'ajustement
\begin{equation*}
  C_{i,j+1} = \lambda_j C_{i,j} + \alpha_j.
\end{equation*}
Ainsi, en plus de considérer une tendance multiplicative entre les
périodes (les facteurs $\lambda_j$), on suppose qu'il y a aussi une
tendance additive (les facteurs $\alpha_j$). Si la tendance
incrémentale est nulle, il est possible d'obtenir la méthode
Chain-Ladder en choisissant une méthode d'estimation appropriée.

La \autoref{fig:deterministe:H2} représente le nuage de points et la
droite entre les périodes $j$ et $j+1$, ici la transition entre une
période 2 et une période 3.

\begin{figure}
  \centering
  \includegraphics{images/H2}
  \caption{Droite des moindres carrés pour le nuage de points entre
    les périodes 2 et 3}
  \label{fig:deterministe:H2}
\end{figure}

Pour déterminer les paramètres de la droite, on utilise les moindres
carrés. Ainsi, on veut trouver les paramètres $\hat{\lambda}_j$ et
$\hat{\alpha}_j$ qui minimisent:
\begin{equation*}
  Q = \sum_{i=1}^{n-k} (C_{i,k+1} - \alpha_k - \lambda_k C_{i,k} )^2.
\end{equation*}
On a alors
\begin{align*}
  \frac{\delta Q}{\delta \alpha_k} &= \sum_{i=1}^{n-k} (C_{i,k+1} - \alpha_k - \lambda_k C_{i,k} ) = 0 \\
  \frac{\delta Q}{\delta \lambda_k} &= \sum_{i=1}^{n-k} C_{i,k} (C_{i,k+1} - \alpha_k - \lambda_k C_{i,k} ) = 0.
\end{align*}

\begin{align}
  \sum_{i=1}^{n-k} C_{i,k+1} - (n-k) \alpha_k - \lambda_k
  \sum_{i=1}^{n-k}  C_{i,k}
  &= 0 \label{eq:deterministe:eq1res} \\
  \sum_{i=1}^{n-k} C_{i,k} C_{i,k+1} -  \alpha_k \sum_{i=1}^{n-k}
  C_{i,k} - \lambda_k \sum_{i=1}^{n-k}  C_{i,k}^2
  &= 0. \label{eq:deterministe:eq2res}
\end{align}
En effectuant \eqref{eq:deterministe:eq2res} -
$\frac{\sum_{i=1}^{n-k} C_{i,k}}{n-k} \times$
\eqref{eq:deterministe:eq1res}, on obtient
\begin{align*}
  \frac{1}{n-k} \sum_{i=1}^{n-k} C_{i,k} C_{i,k+1} - \overline{C}_{k+1}^{(k)} \overline{C}_{k}^{(k)}
  &= \lambda_k \left(\frac{1}{n-k} \sum_{i=1}^{n-k}  C_{i,k}^2 - \overline{C}_{k}^{(k)} \overline{C}_{k}^{(k)} \right)
\end{align*}
\begin{align*}
  \hat{\lambda}_k &= \frac{\frac{1}{n-k} \sum_{i=1}^{n-k} C_{i,k} C_{i,k+1} - \overline{C}_{k+1}^{(k)} \overline{C}_{k}^{(k)} }
                    {\frac{1}{n-k} \sum_{i=1}^{n-k}  C_{i,k}^2 - \overline{C}_{k}^{(k)2} }
\end{align*}
avec
\begin{align*}
  \overline{C}_{k}^{(k)} &= \frac{1}{n-k}  \sum_{i=1}^{n-k} C_{i,k} \\
  \overline{C}_{k+1}^{(k)} &= \frac{1}{n-k} \sum_{i=1}^{n-k} C_{i,k+1}
\end{align*}
alors que
\begin{equation*}
  \hat{\alpha}_k = \overline{C}_{k+1}^{(k)} - \hat{\lambda}_k \overline{C}_{k}^{(k)}.
\end{equation*}

On note que $\alpha_{n}$ ne peut être calculé car on cherche à la fois
une tendance multiplicative et additive alors qu'il n'y a qu'un seul
couple d'observations. La convention veut que l'on ne considère que
l'effet multiplicatif pour cette dernière période de développement. Il
est généralement préférable d'utiliser la méthode London Chain
seulement s'il y a de bonnes raisons de croire qu'il y a un facteur
additif en plus d'un facteur multiplicatif dans le modèle.

\begin{exemple}
  Estimer les provisions avec la méthode London Chain
  \begin{center}
    \begin{tabular}{|l|l l l l l|}\hline
      Année & $1$ & $2$ & $3$ & $4$ & $5$  \\ \hline
      1997 &$\nombre{26312}$&	$\nombre{57779}$&	$\nombre{82451}$&	$\nombre{95506}$&	$\nombre{101604}$\\
      1998 &$\nombre{30470}$&	$\nombre{65482}$&	$\nombre{90973}$&	$\nombre{103562}$&	\\
      1999 &$\nombre{49756}$&	$\nombre{101587}$&	$\nombre{136854}$&	&	\\
      2000 &$\nombre{50420}$&	$\nombre{102735}$&	& &	\\
      2001 &$\nombre{56762}$&	&&&\\ \hline
    \end{tabular}
  \end{center}

  Évidemment, l'évaluation s'effectue beaucoup plus rapidement par
  ordinateur\dots\ On calcule $\alpha_1$ et $\lambda_1$ pour
  l'exemple:
  \begin{align*}
    \overline{C}_{1}^{(1)}
    &= \frac{1}{5-1}  \sum_{i=1}^{5-1} C_{i,1} \\
    &= \frac{\nombre{26312} +\nombre{30470}
      +\nombre{49756}+\nombre{50420}}{4} = \nombre{39239,5} \\
    \overline{C}_{1+1}^{(1)}
    &= \frac{1}{5-1} \sum_{i=1}^{5-1} C_{i,1+1} \\
    &= \frac{\nombre{57779}+\nombre{65482}+\nombre{101587}+\nombre{102735}}{4} = \nombre{81895,75}.
  \end{align*}
  \begin{align*}
    \hat{\lambda}_1
    &= \frac{\frac{1}{5-1} \sum_{i=1}^{5-1} C_{i,1} C_{i,1+1} - \overline{C}_{1+1}^{(1)} \overline{C}_{1}^{(1)} }
      {\frac{1}{5-1} \sum_{i=1}^{5-1}  C_{i,1}^2 - \overline{C}_{1}^{(1)2} }   \\
    &=   \frac{\frac{1}{4}(\nombre{26312}*\nombre{57779} + \nombre{30470}*\nombre{65482} + \nombre{49756}*\nombre{101587} + \nombre{50420}*\nombre{102735})  - \nombre{39239,5}*\nombre{81895,75} }
                      {\frac{1}{4} (\nombre{26312}^2+ \nombre{30470}^2 + \nombre{49756}^2 + \nombre{50420}^2) - \nombre{39239,5}^2}\\
    &= 1,86768.
  \end{align*}
  \begin{align*}
    \hat{\alpha}_1
    &= \overline{C}_{1+1}^{(1)} - \hat{\lambda}_1 \overline{C}_{1}^{(1)} \\
    &= \nombre{81895,75} - 1,86768 * \nombre{39239,5} \\
    &= \nombre{8608,88}.
  \end{align*}
  Au final, on obtient les résultats suivants:
  \begin{center}
    \begin{tabular}{|l|l l l l|}\hline
      k & $1 \rightarrow 2$ & $2 \rightarrow 3$ & $3 \rightarrow 4$ & $4 \rightarrow 5$   \\ \hline
      $\lambda_k$ & $1,86768$  & $1,3533$ & $1,1469$ & $1,0638$\\
      $\alpha_k$  & $\nombre{8608,92}$ & $\nombre{1495,52}$ & $42,21$ &\\ \hline
    \end{tabular}
  \end{center}
  Donc, si on complète le triangle, la dernière ligne est
  \begin{align*}
    \widehat{C}_{5,2}
    &= \nombre{8608,92} + 1,86768*\nombre{53762} = \nombre{109019} \\
    \widehat{C}_{5,3}
    &= \nombre{1495,52} + 1,3533*\nombre{109019} = \nombre{149031}\\
    & \dots
  \end{align*}
  \qed
\end{exemple}


\section{Méthode des provisions constituées}
\label{sec:deterministe:provisions-constituees}

Pour cette méthode, deux facteurs de projection sont utilisés: un pour
les paiements et un pour les provisions. La méthode Chain-Ladder
classique sur l'encouru total (qui correspond aux provisions
individuelles et aux paiements) suppose que les paiements et les
provisions se développent de manière identique. La méthode des
provisions constituées utilise davantage d'informations que le modèle
Chain-Ladder classique et est utile pour les branches à développement
très lentes, ou lorsque très peu de sinistres sont réglés la première
année.

\begin{description}
\item[Modèle pour les provisions] On note
  \begin{itemize}
  \item $Q_{i,j}$ est la provision pour les sinistres survenus au
    cours de l'année $i$, inscrite au passif du bilan en fin d'année
    $i+j-1$.
  \end{itemize}
  Le modèle est
  \begin{equation*}
    Q_{i, j+1} = k_{j+1} Q_{ij} - Y_{i, j+1},
  \end{equation*}
  où $k_{j+1}$ mesure la variation entre les années $j$ et $j+1$ de la
  prévision faite sur le coût total de sinistres survenus à l'année
  $i$.

  On estime $k_{j+1}$ :
  \begin{equation*}
    \hat{k}_{j+1} = \frac{\sum_{i=1}^{n-j} \left( Y_{i j+1} + Q_{i j+1} \right)}{\sum_{i=1}^{n-j} Q_{i j}}.
  \end{equation*}
  %
\item[Modèle pour les paiements] Le montant $Y_{i, j+1}$ payé au cours
  de l'année de développement $j+1$ est une fraction de $ Q_{ij}$:
  \begin{equation*}
    Y_{i, j+1} = h_{j+1}  Q_{ij}.
  \end{equation*}
  On estime $h_{j+1}$
  \begin{equation*}
    \hat{h}_{j+1} = \frac{\sum_{i=1}^{n-j} Y_{i j+1}}{\sum_{i=1}^{n-j} Q_{i j}}
  \end{equation*}
  %
\item[Extrapolation des triangles] Le triangle des paiements et des
  provisions sont complétés simultanément, diagonale par diagonale, en
  utilisant le formules précédentes l'une après l'autre.
  \begin{enumerate}
  \item On commence par la première diagonale inconnue des paiements:
    \begin{equation*}
      \hat{Y}_{i, n-i+2} = \hat{h}_{n-i+2}  Q_{i n-i+1} \text{ pour } i=2,\ldots,n.
    \end{equation*}
  \item On complète la première diagonale inconnue des provisions:
    \begin{equation*}
      \hat{Q}_{i, n-i+2} = k_{n-i+2} Q_{i n-i+1} - \hat{Y}_{i, n-i+2}
      \text{ pour } i=2, \ldots, n
    \end{equation*}
  \item On commence par l'autre diagonale inconnue des paiements.
  \item $\ldots$
  \end{enumerate}
\end{description}

On peut combiner ce modèle avec le modèle classique Chain-Ladder. En
combinant les équations des provisions et des paiements, on obtient
\begin{align*}
  Q_{i, j+1}
  &= k_{j+1} Q_{ij} - Y_{i, j+1}\\
  &= k_{j+1} Q_{ij} - h_{j+1}  Q_{ij} \\
  &= (k_{j+1} - h_{j+1}) Q_{ij} \\
  &= \left(\frac{\sum_{i=1}^{n-j} \left( Y_{i j+1} + Q_{i j+1} \right)}{\sum_{i=1}^{n-j} Q_{i j}}
    - \frac{\sum_{i=1}^{n-j} Y_{i j+1}}{\sum_{i=1}^{n-j} Q_{i j}} \right)  Q_{ij} \\
  &= \frac{\sum_{i=1}^{n-j} Q_{i j+1} }{\sum_{i=1}^{n-j} Q_{i j}}  Q_{ij},
\end{align*}
ce qui revient à la méthode Chain-Ladder. Cette méthode utilise plus
d'informations que la méthode Chain-Ladder, on peut donc choisir des
facteurs de développements différents pour les paiements et les
provisions (qui, elles, sont plus subjectives que les paiements - raison
fiscales, politiques, etc.).

\begin{exemple}
  Compléter les triangles de paiements et de provisions suivants:
  \begin{center}
    \begin{tabular}{|l|l l l l l|}\hline
      Année & $1$ & $2$ & $3$ & $4$ & $5$  \\ \hline
      1997 &$15,40$& $4,90$& $7,77$& $7,19$& $4,3$\\
      1998 &$16,61$& $2,60$& $11,03$& $9,12$& \\
      1999 &$21,35$& $7,29$& $5,59$& &\\
      2000 &$24,52$& $8,49$& & &\\
      2001 &$30,47$& & & &\\ \hline
    \end{tabular}
  \end{center}

  \begin{center}
    \begin{tabular}{|l|l l l l l|}\hline
      Année & $1$ & $2$ & $3$ & $4$ & $5$  \\ \hline
      1997 &$20,0$& $17,39$& $11,06$& $4,50$& $0,60$ \\
      1998 &$22,0$& $22,40$& $13,13$& $5,20$& \\
      1999 &$22,5$& $18,66$& $15,22$& & \\
      2000 &$25,0$& $20,32$& & &\\
      2001 &$25,0$&      & & & \\ \hline
    \end{tabular}
  \end{center}

  On a
  \begin{align*}
    \hat{k}_{1+1}
    &= \frac{\sum_{i=1}^{5-1} \left( Y_{i 1+1} + Q_{i 1+1} \right)}{\sum_{i=1}^{5-1} Q_{i 1}} \\
    &= \frac{17,39+4,90+22,4+2,6+18,66+7,29+20,32+8,49}{20,0+22,0+22,5+25,0} \\
    &= 1,1402\\
    \hat{k}_{3} &= 1,0915\\
    \hat{k}_{4} &= 1,0752\\
    \hat{k}_{5} &= 1,0889
  \end{align*}
  et
  \begin{align*}
    \hat{h}_{1+1}
    &= \frac{\sum_{i=1}^{5-1} Y_{i 1+1}}{\sum_{i=1}^{5-1} Q_{i 1}} \\
    &= \frac{4,90+2,6+7,29+8,49}{20,0+22,0+22,5+25,0} \\
    &= 0,2601 \\
    \hat{h}_{3} &= 0,4173\\
    \hat{h}_{4} &= 0,6742\\
    \hat{h}_{5} &= 0,9556.
  \end{align*}
  Et donc:
  \begin{center}
    \begin{tabular}{|l|l l l l l|}\hline
      Année & $1$ & $2$ & $3$ & $4$ & $5$  \\ \hline
      1997 &$15,40$& $4,90$& $7,77$& $7,19$& $4,3$\\
      1998 &$16,61$& $2,60$& $11,03$& $9,12$& $4,97$\\
      1999 &$21,35$& $7,29$& $5,59$& $10,26$& $5,83$\\
      2000 &$24,52$& $8,49$& $8,48$& $9,24$ &$5,25$\\
      2001 &$30,47$& $6,50$& $9,18$& $10,00$&$5,68$\\ \hline
    \end{tabular}
  \end{center}

  \begin{center}
    \begin{tabular}{|l|l l l l l|}\hline
      Année & $1$ & $2$ & $3$ & $4$ & $5$  \\ \hline
      1997 &$20,0$& $17,39$& $11,06$& $4,50$& $0,60$ \\
      1998 &$22,0$& $22,40$& $13,13$& $5,20$& $0,69$\\
      1999 &$22,5$& $18,66$& $15,22$& $6,10$& $0,81$\\
      2000 &$25,0$& $20,32$& $13,70$& $5,49$& $0,73$\\
      2001 &$25,0$& $22,0$ & $14,84$& $5,95$& $0,79$\\ \hline
    \end{tabular}
  \end{center}
  En sommant les provisions et les paiements, on obtient les encourus
  totaux. %
  \qed
\end{exemple}


\section{Méthode des moindres carrés de DeVylder}
\label{sec:deterministe:devylder}

Cette méthode est basée sur les incréments et non plus les montants
cumulatifs. Elle repose sur l'équation:
\begin{equation*}
  Y_{i,j} = r_j p_i,
\end{equation*}
où
\begin{itemize}
\item $p_i$ est la charge ultime des sinistres survenus l'années $i$;
  et
\item $r_j$ est la proportion du montant $p_i$ payé dans l'année de
  déroulement $j$.
\end{itemize}
Le triangle peut s'exprimer comme:

\begin{center}
  \begin{tabular}{|l l l l l|}\hline
    $r_1 p_1$ & $r_2 p_1$ & $\ldots$ & $r_{n-1} p_1$ & $r_n p_1$ \\
    $r_1 p_2$ & $r_2 p_2$ & $\ldots$ & $r_{n-1} p_2$ & \\
    $\ldots$ & $\ldots$ & $\ldots$ &  & \\
    $r_1 p_{n-1}$ & $r_2 p_{n-1}$ & &  & \\
    $r_1 p_n$ & & &  & \\ \hline
  \end{tabular}
\end{center}

Afin d'obtenir des estimateurs pour les paramètres inconnus, on
minimise la somme des carrés des écarts entre les valeurs observées
$Y_{ij}$ et leur forme théorique $r_j p_i$:
\begin{equation*}
  \sum_{i+j \le n } (Y_{i,j} - r_j p_i)^2
\end{equation*}
avec contrainte $r_1 + r_2 + \ldots + r_n = 1$. On obtient
\begin{align*}
  \hat{p}_{i} &= \frac{\sum_{j} \hat{r}_{j} Y_{i j}}{\sum_{j} \hat{r}_{j}^2 }\\
  \hat{r}_{j} &= \frac{\sum_{i} \hat{p}_{i} Y_{i j}}{\sum_{i} \hat{p}_{i}^2 }
\end{align*}
qu'il faut résoudre numériquement de manière récursive. Afin de
s'assurer que $r_1 + r_2 + \ldots + r_n = 1$, après chaque itération,
on peut diviser chaque nouveau $\hat{r}_{j}$ par la somme des nouveaux
$\hat{r}_{j}$.

\begin{exemple}
  On considère le triangle de paiement suivant:
  \begin{center}
    \begin{tabular}{|l|l l l l l|}\hline
      Année & $1$ & $2$ & $3$ & $4$ & $5$  \\ \hline
      1 (1997)& $\nombre{26312}$ & $\nombre{31467}$ & $\nombre{24672}$ & $\nombre{13055}$ & $\nombre{6158}$ \\
      2 (1998)& $\nombre{30470}$ & $\nombre{35012}$ & $\nombre{25491}$ & $\nombre{12589}$ &  \\
      3 (1999)& $\nombre{49756}$ & $\nombre{51831}$ & $\nombre{35267}$ &&\\
      4 (2000)& $\nombre{50420}$ & $\nombre{52315}$ & &&\\
      5 (2001)& $\nombre{56762}$ &  &&&\\ \hline
    \end{tabular}
  \end{center}
  Trouver la provision totale, si on a estimé que les $\hat{r}_{j}$
  (après convergence de l'algorithme) sont égaux à:
  \begin{center}
    \begin{tabular}{|l l l l l|}\hline
      $\hat{r}_1$ & $\hat{r}_2$ & $\hat{r}_3$ & $\hat{r}_4$ & $\hat{r}_5$ \\ \hline
      $0,2875$ &$0,3086$ &$0,2221$ &$0,1210$ &$0,0606$ \\ \hline
    \end{tabular}
  \end{center}
  Les $\hat{r}_{j}$ correspondent aux proportions annuelles payées de
  l'encouru total.

  \begin{align*}
    \hat{p}_{1} &= \frac{\sum_{j} \hat{r}_{j} Y_{1 j}}{\sum_{j} \hat{r}_{j}^2 }\\
                &= \frac{\nombre{26312}*0,2875 +\nombre{31467}*0,3086 + \nombre{24672}*0,2221 + \nombre{13055}*0,1210 + \nombre{6158}*0,0606}
                  {0,2875^2 + 0,3086^2 + 0,2221^2 + 0,1210^2 + 0,0606^2} \\
                &= \nombre{100603,22} \\
    \hat{p}_{2} &= \nombre{110585,90} \\
    \hat{p}_{3} &= \nombre{167806,52} \\
    \hat{p}_{4} &= \nombre{172235,48} \\
    \hat{p}_{5} &= \nombre{197480,59}.
  \end{align*}
  Donc, la provision revient à être $\hat{p}_j - C_{5-j+1,j}$!
  \begin{align*}
    \hat{p}_1 - C_{5,j} &= \nombre{101603,22} - \nombre{101604} \\
    \hat{p}_2 - C_{4,j} &= \nombre{110585,90} - \nombre{103562} \\
    \hat{p}_3 - C_{3,j} &= \nombre{167806,52} - \nombre{136854} \\
    \hat{p}_4 - C_{2,j} &= \nombre{172235,48} - \nombre{102735} \\
    \hat{p}_5 - C_{1,j} &= \nombre{197480,59} - \nombre{56762}.
  \end{align*}
  \qed
\end{exemple}


\section{Exercices}
\label{sec:prov:exercices}

\Opensolutionfile{reponses}[reponses-deterministe]
\Opensolutionfile{solutions}[solutions-deterministe]

\begin{Filesave}{reponses}
\bigskip
\section*{Réponses}

\end{Filesave}

\begin{Filesave}{solutions}
\section*{Chapitre \ref*{chap:deterministe}}
\addcontentsline{toc}{section}{Chapitre \protect\ref*{chap:deterministe}}

\end{Filesave}

\begin{exercice}
  Le \autoref{tab:deterministe:loss1} présente les différents
  paiements réalisés par l'assureur YTR.
  \begin{table}[!h]
    \centering
    \caption{Paiements réalisés par l'assureur YTR}
    \label{tab:deterministe:loss1}
    \begin{tabular}{cccc}
      \toprule
      Date paiement & Numéro assuré & Date survenance & Montant\\
      \midrule
      04/2000 & 456 & 03/2000 & $200$\\
      09/2000 & 476 & 08/2000 & $225$\\
      02/2001 & 456 & 03/2000 & $40$\\
      10/2001 & 476 & 08/2000 & $57$\\
      01/2002 & 456 & 03/2000 & $90$\\
      04/2003 & 476 & 08/2000 & $102$\\
      02/2004 & 476 & 08/2000 & $16$\\
      10/2001 & 287 & 10/2001 & $532$\\
      12/2002 & 287 & 10/2001 & $125$\\
      02/2003 & 937 & 03/2001 & $57$\\
      01/2004 & 287 & 10/2001 & $18$\\
      05/2002 & 456 & 03/2002 & $717$\\
      08/2003 & 456 & 03/2002 & $13$\\
      04/2004 & 456 & 03/2002 & $72$\\
      07/2003 & 101 & 07/2003 & $440$\\
      01/2004 & 867 & 03/2003 & $120$\\
      04/2004 & 200 & 02/2004 & $400$\\
      10/2004 & 956 & 08/2004 & $220$\\
      \bottomrule
    \end{tabular}
  \end{table}
  \begin{enumerate}
  \item Construire les triangles des paiements cumulés.
  \item Calculer les différents facteurs de développement. Discuter.
  \item Quel est le nombre de périodes nécessaires pour observer le
    développement complet des paiements?
  \item Déterminer le montant de provision nécessaire en utilisant la
    méthode Chain-Ladder.
  \item Un facteur de développement peut-il être inférieur à $1$?
  \end{enumerate}
  \begin{rep}
    \begin{enumerate}
      \stepcounter{enumi}
    \item $\lambda_1 = 1,167928$, $\lambda_2 = 1,114720$,
      $\lambda_3 = 1,090498$, $\lambda_4 = 1,022409$
    \item $5$
    \item $524$
    \item Oui
    \end{enumerate}
  \end{rep}
  \begin{sol}
    \begin{enumerate}
    \item Le triangle des paiements cumulés est présenté dans le
      tableau~\ref{tab:tri1}. Pour la case $(i, j)$, il suffit de
      sommer les montants des paiements pour les sinistres survenus
      pendant l'année~$i$ et payés pendant la $j\ieme$~période après
      la survenance.
      \begin{table}[!h]
        \centering
        \begin{tabular}{cccccc}
          \toprule
          & $1$ & $2$ & $3$ & $4$ & $5$\\
          \midrule
          2000 & $425$ & $522$ & $612$ & $714$ & $730$\\
          2001 & $532$ & $657$ & $714$ & $732$ & -\\
          2002 & $717$ & $730$ & $802$ & - & -\\
          2003 & $440$ & $560$ & - & - & -\\
          2004 & $620$ & - & - & - & -\\
          \bottomrule
        \end{tabular}
        \caption{Paiements réalisés par l'assureur YTR}
        \label{tab:tri1}
      \end{table}
    \item On a
      \begin{align*}
        \lambda_1 &= \frac{522 + 657 + 730 + 560}{425 + 532 + 717 + 440}\\
                  &=  \frac{\nombre{2469}}{\nombre{2114}}\\
                  &= 1,167928\\
        \lambda_2 &= \frac{\nombre{2128}}{\nombre{1909}}\\
                  &= 1,114720\\
        \lambda_3 &= \frac{\nombre{1446}}{\nombre{1326}}\\
                  &= 1,090498\\
        \lambda_4 &= \frac{\nombre{730}}{\nombre{714}}\\
                  &= 1,022409.
      \end{align*}
      Avant d'accepter aveuglément ces facteurs de développement, on
      peut analyser les facteurs de développement individuels (case
      par case) tels que présentés dans le tableau~\ref{tab:tri2}.
      Normalement, on devrait observer une certaine stabilité pour une
      même période de développement ($j$) entre les différentes années
      de survenance ($i$), ce qui n'est pas toujours le cas ici. Par
      exemple, on remarque que le passage de la période $1$ à la
      période $2$ est relativement stable pour les années 2000, 2001
      et 2003 (respectivement $1,23$, $1,23$ et $1,27$), un certain
      ralentissement dans la cadence des paiements est observé pour
      l'année 2002 ($1,02$). Après analyse, l'actuaire devra
      déterminer si ce ralentissement peut s'expliquer par un
      changement dans la politique de la compagnie pour cette année de
      survenance (et alors peut-être retirer les données de 2002 pour
      le calcul des facteurs de développement).
      \begin{table}[!h]
        \centering
        \begin{tabular}{ccccc}
          \toprule
          & $1-2$ & $2-3$ & $3-4$ & $4-5$\\
          \midrule
          2000 & $1,23$ & $1,17$ & $1,17$ & $1,02$\\
          2001 & $1,23$ & $1,09$ & $1,03$ & -\\
          2002 & $1,02$ & $1,10$ & - & -\\
          2003 & $1,27$ & - & - & -\\
          2004 & - & - & - & -\\
          \bottomrule
        \end{tabular}
        \caption{Facteurs de développement individuels}
        \label{tab:tri2}
      \end{table}
    \item Selon les données enregistrées par la compagnie, on observe
      un facteur de $1,02$ entre la $4\ieme$~période et la
      $5\ieme$~période de développement. Cette valeur étant très près
      de $1,00$, il est raisonnable de supposer que les paiements sont
      pratiquement complets après $5$~périodes de développement.
      Idéalement, le fait de pouvoir consulter une base de données
      semblables plus mature permettrait de confirmer ou d'infirmer
      cette hypothèse. Enfin, connaître le type de portefeuille
      (assurance automobile - dommage matériel, assurance automobile -
      dommage corporel, assurance responsabilité professionnelle,
      etc.) permettrait également d'avoir une idée du nombre de
      périodes de développement nécessaires.
    \item Le triangle complété des paiements cumulés est présenté dans
      le tableau~\ref{tab:tri3}. Par exemple,
      \begin{align*}
        (732)(1,022409) &= 748\\
        (802)(1,090498) &= 875\\
        (802)(1,090498)(1,022409) &= 894.
      \end{align*}
      \begin{table}[!h]
        \centering
        \begin{tabular}{cccccc}
          \toprule
          & $1$ & $2$ & $3$ & $4$ & $5$\\
          \midrule
          2000 & $425$ & $522$ & $612$ & $714$ & $730$\\
          2001 & $532$ & $657$ & $714$ & $732$ & $\textcolor{red}{748}$\\
          2002 & $717$ & $730$ & $802$ & $\textcolor{red}{875}$ & $\textcolor{red}{894}$\\
          2003 & $440$ & $560$ & $\textcolor{red}{624}$ & $\textcolor{red}{681}$ & $\textcolor{red}{696}$\\
          2004 & $620$ & $\textcolor{red}{724}$ & $\textcolor{red}{807}$ & $\textcolor{red}{880}$ & $\textcolor{red}{900}$\\
          \bottomrule
        \end{tabular}
        \caption{Paiements réalisés et estimés par l'assureur YTR}
        \label{tab:tri3}
      \end{table}
      Le montant total de la provision est donné par la différence entre
      le montant total des paiements ultimes et le montant total des
      paiements déjà réalisés:
      \begin{align*}
        730 + 748 + 894 + 696 + 900 &= \nombre{3968}\\
        730 + 732 + 802 + 560 + 620 &= \nombre{3444}\\
        \nombre{3968} - \nombre{3444} &= 524.
      \end{align*}
    \item Oui, c'est possible. Cela indiquerait que l'assureur a reçu
      des remboursements (paiements en trop, recours judiciaires,
      etc.).
    \end{enumerate}
  \end{sol}
\end{exercice}

\begin{exercice}
  L'actuaire de la compagnie GFR possède les informations suivantes
  sur les montants payés cumulatifs, en fonction des années de
  développement et de l'année de survenance du sinistre:
  \begin{center}
    \begin{tabular}{|l|l l l l l l l|}\hline
      Année & $1$ & $2$ & $3$ & $4$ & $5$ & $6$ & $7$\\ \hline
      1993 & $\nombre{1780}$ & $\nombre{2673}$ & $\nombre{2874}$ & $\nombre{3094}$ & $\nombre{3157}$ & $\nombre{3166}$ & $\nombre{3166}$ \\
      1994 & $\nombre{3226}$ & $\nombre{4219}$ & $\nombre{4532}$ & $\nombre{4881}$ & $\nombre{5144}$ & $\nombre{5199}$ & \\
      1995 & $\nombre{3652}$ & $\nombre{4989}$ & $\nombre{5762}$ & $\nombre{6436}$ & $\nombre{6720}$ & & \\
      1996 & $\nombre{2723}$ & $\nombre{4301}$ & $\nombre{5526}$ & $\nombre{6231}$ & & & \\
      1997 & $\nombre{2923}$ & $\nombre{4666}$ & $\nombre{5349}$ & & & & \\
      1998 & $\nombre{2990}$ & $\nombre{5417}$ & & & & & \\
      1999 & $\nombre{3917}$ & & & & & &\\ \hline
    \end{tabular}
  \end{center}
  En utilisant diverses variantes de la méthode Chain-Ladder, estimer
  les provisions pour ces données,
  \begin{enumerate}
  \item si on estime les facteurs de déroulement par la méthode
    \emph{Chain Ladder} usuelle;
  \item si on estime les facteurs de déroulement ($\lambda_j$) par une
    moyenne arithmétique;
  \item si on estime les facteurs de déroulement ($\lambda_j$) par une
    moyenne géométrique.
  \end{enumerate}
  \begin{rep}
    \begin{enumerate}
    \item $\nombre{7089}$
    \item $\nombre{6854}$
    \item $\nombre{6767}$
    \end{enumerate}
  \end{rep}
  \begin{sol}
    \begin{enumerate}
    \item On a
      \begin{align*}
        \lambda_1 &= \frac{\nombre{2673} + \nombre{4219} + \nombre{4989}
                    +\nombre{4301} + \nombre{4666} + \nombre{5417}}{\nombre{1780} +
                    \nombre{3226} + \nombre{3652} + \nombre{2723} + \nombre{2923} + \nombre{2990}}\\
                  &=  \frac{\nombre{26265}}{\nombre{17294}}\\
                  &= 1,518735\\
        \lambda_2 &=  1,153252\\
        \lambda_3 &= 1,104205\\
        \lambda_4 &=  1,042329\\
        \lambda_5 &= 1,007710\\
        \lambda_6 &= 1,000000.
      \end{align*}
      Le triangle complété est
      \begin{center}
        \begin{tabular}{|l|l l l l l l l|}\hline
          Année & $1$ & $2$ & $3$ & $4$ & $5$ & $6$ & $7$\\ \hline
          1993& $\nombre{1780}$& $\nombre{2673}$& $\nombre{2874}$& $\nombre{3094}$& $\nombre{3157}$& $\nombre{3166}$& $\nombre{3166}$ \\
          1994& $\nombre{3226}$& $\nombre{4219}$& $\nombre{4532}$& $\nombre{4881}$& $\nombre{5144}$& $\nombre{5199}$& $\nombre{5199}$ \\
          1995& $\nombre{3652}$& $\nombre{4989}$& $\nombre{5762}$& $\nombre{6436}$& $\nombre{6720}$& $\nombre{6772}$& $\nombre{6772}$ \\
          1996& $\nombre{2723}$& $\nombre{4301}$& $\nombre{5526}$& $\nombre{6231}$& $\nombre{6495}$& $\nombre{6545}$& $\nombre{6545}$ \\
          1997& $\nombre{2923}$& $\nombre{4666}$& $\nombre{5349}$& $\nombre{5906}$& $\nombre{6156}$& $\nombre{6204}$& $\nombre{6204}$ \\
          1998& $\nombre{2990}$& $\nombre{5417}$& $\nombre{6247}$& $\nombre{6898}$& $\nombre{7190}$& $\nombre{7246}$& $\nombre{7246}$ \\
          1999& $\nombre{3917}$& $\nombre{5949}$& $\nombre{6861}$& $\nombre{7575}$& $\nombre{7896}$& $\nombre{7957}$& $\nombre{7957}$\\\hline
        \end{tabular}
      \end{center}
      Le montant ultime payé est la somme de la dernière colonne,
      $\nombre{43088}$, et le montant de la provision est la
      différence entre le montant ultime et le montant total payé
      (somme de la diagonale principale):
      $\nombre{43088} - \nombre{35999} = \nombre{7089}$.
    \item On calcule, pour chaque cellule, le facteur de développement
      \begin{center}
        \begin{tabular}{|l|l l l l l l|}\hline
          Année & 1$\rightarrow$ 2 &  2$\rightarrow$ 3 &  3$\rightarrow$ 4 &
                                                                             4$\rightarrow$ 5 & 5$\rightarrow$ 6 & 6$\rightarrow$ 7 \\ \hline
          1993 &$1,501685$&$1,075196$&$1,076548$&$1,020362$&$1,002851$ & $1$\\
          1994 &$1,307812$&$1,074188$&$1,077008$&$1,053882$&$1,010692$ & $$\\
          1995 &$1,366101$&$1,154941$&$1,116973$&$1,044127$&$$ & $$\\
          1996 &$1,579508$&$1,284817$&$1,127579$& & & \\
          1997 &$1,596305$&$1,146378$& & & & \\
          1998 &$1,811706$& &&& & \\\hline
          Moy. arith. &$1,527186$&$1,147104$&$1,099527$&$1,039457$ & $1,006771$ & $1$\\\hline
        \end{tabular}
      \end{center}
      Le triangle complété est
      \begin{center}
        \begin{tabular}{|l|l l l l l l l|}\hline
          Année & $1$ & $2$ & $3$ & $4$ & $5$ & $6$ & $7$\\ \hline
          1993& $\nombre{1780}$& $\nombre{2673}$& $\nombre{2874}$& $\nombre{3094}$& $\nombre{3157}$& $\nombre{3166}$& $\nombre{3166}$ \\
          1994& $\nombre{3226}$& $\nombre{4219}$& $\nombre{4532}$& $\nombre{4881}$& $\nombre{5144}$& $\nombre{5199}$& $\nombre{5199}$ \\
          1995& $\nombre{3652}$& $\nombre{4989}$& $\nombre{5762}$& $\nombre{6436}$& $\nombre{6720}$& $\nombre{6766}$& $\nombre{6766}$ \\
          1996& $\nombre{2723}$& $\nombre{4301}$& $\nombre{5526}$& $\nombre{6231}$& $\nombre{6477}$& $\nombre{6521}$& $\nombre{6521}$ \\
          1997& $\nombre{2923}$& $\nombre{4666}$& $\nombre{5349}$& $\nombre{5881}$& $\nombre{6113}$& $\nombre{6155}$& $\nombre{6155}$ \\
          1998& $\nombre{2990}$& $\nombre{5417}$& $\nombre{6214}$& $\nombre{6832}$& $\nombre{7102}$& $\nombre{7150}$& $\nombre{7150}$ \\
          1999& $\nombre{3917}$& $\nombre{5982}$& $\nombre{6862}$& $\nombre{7545}$& $\nombre{7843}$& $\nombre{7896}$& $\nombre{7896}$\\\hline
        \end{tabular}
      \end{center}
      Le montant ultime payé est la somme de la dernière colonne,
      $\nombre{42853}$, et le montant de la provision est la
      différence entre le montant ultime et le montant total payé
      (somme de la diagonale principale): $\nombre{42853} -
      \nombre{35999} = \nombre{6854}$.
    \item On calcule, pour chaque cellule, le facteur de développement
      \begin{center}
        \begin{tabular}{|l|l l l l l l|}\hline
          Année & 1$\rightarrow$ 2 &  2$\rightarrow$ 3 &  3$\rightarrow$ 4 &
                                                                             4$\rightarrow$ 5 & 5$\rightarrow$ 6 & 6$\rightarrow$ 7 \\ \hline
          1993 &$1,501685$&$1,075196$&$1,076548$&$1,020362$&$1,002851$ & $1$\\
          1994 &$1,307812$&$1,074188$&$1,077008$&$1,053882$&$1,010692$ & $$\\
          1995 &$1,366101$&$1,154941$&$1,116973$&$1,044127$&$$ & $$\\
          1996 &$1,579508$&$1,284817$&$1,127579$& & & \\
          1997 &$1,596305$&$1,146378$& & & & \\
          1998 &$1,811706$& &&& & \\\hline
          Moy. géo. &$1,518409$&$1,144615$&$1,099286$&$1,039361$ & $1,006764$ & $1$\\\hline
        \end{tabular}
      \end{center}
      Le triangle complété est
      \begin{center}
        \begin{tabular}{|l|l l l l l l l|}\hline
          Année & $1$ & $2$ & $3$ & $4$ & $5$ & $6$ & $7$\\ \hline
          1993& $\nombre{1780}$& $\nombre{2673}$& $\nombre{2874}$& $\nombre{3094}$& $\nombre{3157}$& $\nombre{3166}$& $\nombre{3166}$ \\
          1994& $\nombre{3226}$& $\nombre{4219}$& $\nombre{4532}$& $\nombre{4881}$& $\nombre{5144}$& $\nombre{5199}$& $\nombre{5199}$ \\
          1995& $\nombre{3652}$& $\nombre{4989}$& $\nombre{5762}$& $\nombre{6436}$& $\nombre{6720}$& $\nombre{6765}$& $\nombre{6765}$ \\
          1996& $\nombre{2723}$& $\nombre{4301}$& $\nombre{5526}$& $\nombre{6231}$& $\nombre{6476}$& $\nombre{6520}$& $\nombre{6520}$ \\
          1997& $\nombre{2923}$& $\nombre{4666}$& $\nombre{5349}$& $\nombre{5880}$& $\nombre{6112}$& $\nombre{6153}$& $\nombre{6153}$ \\
          1998& $\nombre{2990}$& $\nombre{5417}$& $\nombre{6200}$& $\nombre{6816}$& $\nombre{7084}$& $\nombre{7132}$& $\nombre{7132}$ \\
          1999& $\nombre{3917}$& $\nombre{5948}$& $\nombre{6808}$& $\nombre{7484}$& $\nombre{7778}$& $\nombre{7831}$& $\nombre{7831}$\\\hline
        \end{tabular}
      \end{center}
      Le montant ultime payé est la somme de la dernière colonne,
      $\nombre{42766}$, et le montant de la provision est la
      différence entre le montant ultime et le montant total payé
      (somme de la diagonale principale):
      $\nombre{42766} - \nombre{35999} = \nombre{6767}$.
    \end{enumerate}
  \end{sol}
\end{exercice}

\begin{exercice}
  L'actuaire de la compagnie \emph{Dujardin et cie} a collecté les
  informations suivantes sur les montants payés par année $Y_{i,j}$,
  en fonction des années de développement et de l'année de survenance
  du sinistre
  \begin{center}
    \begin{tabular}{|c|c c c c c c|}\hline
      Année & $1$ & $2$ & $3$ & $4$ & $5$ & $6$ \\ \hline
      1994 & $192$ & $251$ & $153$ & $145$ & $98$  & $0$ \\
      1995 & $205$ & $280$ & $195$ & $150$ & $102$ &   \\
      1996 & $230$ & $345$ & $230$ & $212$ &     &   \\
      1997 & $288$ & $410$ & $275$ &     &     &   \\
      1998 & $398$ & $563$ &     &     &     &   \\
      1999 & $530$ &     &     &     &     &   \\ \hline
    \end{tabular}
  \end{center}
  Estimer les provisions pour ces données si on estime les facteurs de
  déroulement par la méthode Chain-Ladder usuelle.
  \begin{rep}
    $\nombre{3382}$
  \end{rep}
  \begin{sol}
    On construit premièrement le triangle des valeurs cumulées
    \begin{center}
      \begin{tabular}{|c|c c c c c c|}\hline
        Année & $1$ & $2$ & $3$ & $4$ & $5$ & $6$ \\ \hline
        1994 & $192$ & $443$ & $596$ & $741$ & $839$  & $839$ \\
        1995 & $205$ & $485$ & $680$ & $830$ & $932$ &   \\
        1996 & $230$ & $575$ & $805$ & $\nombre{1017}$ &     &   \\
        1997 & $288$ & $698$ & $973$ &     &     &   \\
        1998 & $398$ & $961$ &     &     &     &   \\
        1999 & $530$ &     &     &     &     &   \\ \hline
      \end{tabular}
    \end{center}
    On a
    \begin{align*}
      \lambda_1 &= \frac{\nombre{443} + \nombre{485} + \nombre{575}
                  +\nombre{698} + \nombre{961}}{\nombre{192} +
                  \nombre{205} + \nombre{230} + \nombre{288} + \nombre{398}}\\
                &=  \frac{\nombre{3162}}{\nombre{1313}}\\
                &= 2,408225\\
      \lambda_2 &=  1,387551\\
      \lambda_3 &= 1,243633\\
      \lambda_4 &=  1,127307\\
      \lambda_5 &= 1,000000.
    \end{align*}
    Le triangle complété est
    \begin{center}
      \begin{tabular}{|c|c c c c c c|}\hline
        Année & $1$ & $2$ & $3$ & $4$ & $5$ & $6$ \\ \hline
        1994 & $192$ & $443$ & $596$ & $741$ & $839$  & $839$ \\
        1995 & $205$ & $485$ & $680$ & $830$ & $932$ & $932$  \\
        1996 & $230$ & $575$ & $805$ & $\nombre{1017}$ & $\nombre{1146}$    & $\nombre{1146}$  \\
        1997 & $288$ & $698$ & $973$ &  $\nombre{1210}$   &   $\nombre{1364}$  & $\nombre{1364}$  \\
        1998 & $398$ & $961$ &  $\nombre{1333}$   &  $\nombre{1658}$   & $\nombre{1869}$    & $\nombre{1869}$  \\
        1999 & $530$ &  $\nombre{1276}$   &  $\nombre{1771}$   &   $\nombre{2202}$  &  $\nombre{2483}$   &  $\nombre{2483}$ \\ \hline
      \end{tabular}
    \end{center}
    Le montant ultime payé est la somme de la dernière colonne,
    $\nombre{8634}$, et le montant de la provision est la différence
    entre le montant ultime et le montant total payé (somme de la
    diagonale principale):
    $\nombre{8634} - \nombre{5252} = \nombre{3382}$.
  \end{sol}
\end{exercice}

\begin{exercice}
  Pour une certaine années d'accident, on a les informations
  suivantes:
  \begin{itemize}
  \item primes acquises : $\nombre{1000}$~\$;
  \item rapport Sinistres/Primes espéré : $0,650$;
  \item $\prod_{k=2}^{ult} \lambda_k = 1,12$;
  \item sinistres encourus à ce jour: $600$~\$; et
  \item sinistres payés à ce jour: $500$~\$.
  \end{itemize}
  Calculer l'estimation des montants des provisions en utilisant la
  technique de Bornhuetter-Ferguson.
  \begin{rep}
    $69,64286$
  \end{rep}
  \begin{sol}
    \begin{align*}
      \hat{C}_{i}^{LR} &= \Esp{LR} \times \text{Primes acquises}_i\\
                       &= 0,650 * \nombre{1000} = \nombre{650} \\
      R_i^{BF} &= \hat{C}_{i}^{LR} \times ( 1 - \frac{1}{\prod_{k=2}^{\infty} \lambda_k}) \\
                       &= \nombre{650} (1-\frac{1}{1,12}) = \nombre{69,64286}.
    \end{align*}
  \end{sol}
\end{exercice}

\begin{exercice}
  On suppose qu'on a estimé les facteurs de développement par la
  méthode Chain-Ladder usuelle et qu'on a obtenu
  \begin{center}
    \begin{tabular}{|c c c c c|}\hline
      1$\rightarrow$ 2 &  2$\rightarrow$ 3 &  3$\rightarrow$ 4 &  4$\rightarrow$ 5  &5 $\rightarrow  \infty$  \\ \hline
      $1,75$ & $1,6$ & $1,4$ & $1,1$ & $1,05$ \\ \hline
    \end{tabular}
  \end{center}
  \begin{itemize}
  \item Pour l'année 1998, on estime le 31/12/1999 des paiements
    cumulatifs de $\nombre{320000}$~\$;
  \item les primes acquises de 1998 sont de $\nombre{1000000}$~\$; et
  \item le rapport sinistres/primes espéré de l'année 1998 est de
    $0,650$.
  \end{itemize}
  Estimer les provisions pour l'année d'accident 1998:
  \begin{enumerate}
  \item en utilisant la méthode du rapport sinistres/primes espéré;
  \item en utilisant la méthode Chain-Ladder; et
  \item en utilisant la méthode Bornhuetter-Ferguson.
  \end{enumerate}
  \begin{rep}
    \begin{enumerate}
    \item $\nombre{330000}$
    \item $\nombre{507904}$
    \item $\nombre{398763,1}$
    \end{enumerate}
  \end{rep}
  \begin{sol}
    La première évaluation des provisions de l'année d'accident 1998 est
    le 31 décembre 1998. Ainsi, le 31 décembre 1999, on est à la
    deuxième évaluation.
    \begin{enumerate}
    \item On a
      \begin{align*}
        \underbrace{\hat{C}_{i}^{LR}}_{\text{Pertes ultimes attendues selon LR}} &=
                                                                                   \underbrace{\Esp{LR}}_{\text{Rapport sinistres/primes espéré}} \times \text{Primes acquises}_i\\
                                                                                 &= 0,650 * \nombre{1000000} = \nombre{650000} \\
        \underbrace{R_i^{LR}}_{\text{Provisions selon la méthode du rapport sinistres/primes}}
                                                                                 &= \hat{C}_{i}^{LR} - \underbrace{C_{i, 2}}_{\text{Paiements cumulatifs à la deuxième évaluation}} \\
                                                                                 &= \nombre{650000} - \nombre{320000} = \nombre{330000}.
      \end{align*}

    \item On a
      \begin{align*}
        \underbrace{\hat{C}_{i}^{CL}}_{\text{Pertes ultimes attendues selon CL}} &=
                                                                                   C_{i, 2} \times (\prod_{k=2}^{\infty} \lambda_k)\\
                                                                                 &= \nombre{320000} \times  1,60 * 1,40 * 1,10 * 1,05 \\
                                                                                 &= \nombre{320000} * 2,5872 = \nombre{827904}\\
        \underbrace{R_i^{CL}}_{\text{Provisions selon la méthode CL}}
                                                                                 &= \hat{C}_{i}^{CL} - C_{i, 2} \\
                                                                                 &= \nombre{827904} - \nombre{320000} = \nombre{507904}.
      \end{align*}

    \item On a
      \begin{align*}
        R_i^{BF} &= \hat{C}_{i}^{LR} \times ( 1 - \frac{1}{\prod_{k=2}^{\infty} \lambda_k}) \\
                 &= \nombre{650000} (1-\frac{1}{2,5872}) = \nombre{398763,1}.
      \end{align*}
    \end{enumerate}
  \end{sol}
\end{exercice}

\begin{exercice}
  On suppose qu'on a estimé les facteurs de développement par la
  méthode Chain-Ladder
  \begin{center}
    \begin{tabular}{|c c c c c c|}\hline
      1$\rightarrow$ 2 &  2$\rightarrow$ 3 &  3$\rightarrow$ 4 &  4$\rightarrow$ 5  &5 $\rightarrow$ 6 &6 $\rightarrow  \infty$ \\ \hline
      $1,55$ & $1,5$ & $1,3$ & $1,25$ & $1,15$ & $1,05$ \\ \hline
    \end{tabular}
  \end{center}
  \begin{itemize}
  \item Pour l'année 1999, on estime le 31/12/1999 des paiements
    cumulatifs de $\nombre{120000}$~\$;
  \item les primes acquises de 1999 sont de $\nombre{1350000}$~\$; et
  \item le rapport sinistres/primes espéré de l'année 1999 est de
    $0,600$.
  \end{itemize}
  Estimer les provisions pour l'année d'accident 1999:
  \begin{enumerate}
  \item en utilisant la méthode du rapport sinistres/primes espéré;
  \item en utilisant la méthode Chain-Ladder; et
  \item en utilisant la méthode Bornhuetter Ferguson.
  \item Le 31/12/2002, les paiements cumulatifs sont maintenant de
    $\nombre{200000}$~\$. Pour les trois méthodes précédentes,
    calculer la différence entre la réalisation et la projection.
  \item Quel est l'avantage d'utiliser la méthode de
    Bornhuetter-Ferguson par rapport à celle de Chain-Ladder ?
  \end{enumerate}
  \begin{rep}
    \begin{enumerate}
    \item $\nombre{690000}$
    \item $\nombre{427450,3}$
    \item $\nombre{632449,6}$
    \item $\nombre{162700}$ et $\nombre{73354}$
    \end{enumerate}
  \end{rep}
  \begin{sol}
    \begin{enumerate}
    \item On a
      \begin{align*}
        \underbrace{\hat{C}_{i}^{LR}}_{\text{Pertes ultimes attendues selon LR}} &=
                                                                                   \underbrace{\Esp{LR}}_{\text{Rapport sinistres/primes espéré}} \times \text{Primes acquises}_i\\
                                                                                 &= 0,600 * \nombre{1350000} = \nombre{810000} \\
        \underbrace{R_i^{LR}}_{\text{Provisions selon la méthode du rapport sinistres/primes}}
                                                                                 &= \hat{C}_{i}^{LR} - \underbrace{C_{i, 1}}_{\text{Paiements cumulatifs à la première évaluation}} \\
                                                                                 &= \nombre{810000} - \nombre{120000} = \nombre{690000}.
      \end{align*}

    \item On a
      \begin{align*}
        \underbrace{\hat{C}_{i}^{CL}}_{\text{Pertes ultimes attendues selon CL}} &=
                                                                                   C_{i, 1} \times (\prod_{k=1}^{\infty} \lambda_k)\\
                                                                                 &= \nombre{120000} \times  1,55 * 1,50 * 1,30 * 1,25 * 1,05 \\
                                                                                 &=
                                                                                   \nombre{120000}
                                                                                   *
                                                                                   4,562086
                                                                                   =
                                                                                   \nombre{547450,3}
        \\
        \underbrace{R_i^{CL}}_{\text{Provisions selon la méthode CL}}
                                                                                 &= \hat{C}_{i}^{CL} - C_{i, 1} \\
                                                                                 &= \nombre{547450,3} - \nombre{120000} = \nombre{427450,3}.
      \end{align*}

    \item On a
      \begin{align*}
        R_i^{BF} &= \hat{C}_{i}^{LR} \times ( 1 - \frac{1}{\prod_{k=1}^{\infty} \lambda_k}) \\
                 &= \nombre{810000} (1-\frac{1}{4,562086}) = \nombre{632449,6}.
      \end{align*}
    \item Pour la première méthode, il n'est pas possible de ventiler
      les provisions. Pour la méthode Chain-Ladder, on a
      \begin{align*}
        \nombre{120000} \times 1,55 * 1,50 * 1,30 &= \nombre{362700}\\
        \Delta &= \nombre{362700} - \nombre{200000} = \nombre{162700}.
      \end{align*}
      Pour la méthode BF, on a
      \begin{itemize}
      \item du temps 1 au temps $\infty$, le facteur de développement
        est de $1,55 * 1,50 * 1,30 * 1,25 * 1,15 * 1,05 = 4,562086$;
      \item du temps 2 au temps $\infty$, le facteur de développement
        est de $1,50 * 1,30 * 1,25 * 1,15 * 1,05 = 2,943281$;
      \item du temps 3 au temps $\infty$, le facteur de développement
        est de $1,30 * 1,25 * 1,15 * 1,05 = 1,962187$;
      \item du temps 4 au temps $\infty$, le facteur de développement
        est de $1,25 * 1,15 * 1,05 = 1,509375$;
      \item du temps 5 au temps $\infty$, le facteur de développement
        est de $1,15 * 1,05 = 1,207500$; et
      \item du temps 6 au temps $\infty$, le facteur de développement
        est de $1,05$.
      \end{itemize}
      Ainsi:
      \begin{center}
        \begin{tabular}{|c|c|c|}\hline
          Date ($j$) & $\hat{C}_{i,j}^{BF}$ & $\hat{Y}_{i,j}^{BF}$  \\ \hline
          Décembre 2000 &  $\nombre{217652,72}$ & $\nombre{97652,72}$ \\ \hline
          Décembre 2001 &  $\nombre{355254,3}$ & $\nombre{137601,6}$\\ \hline
          Décembre 2002 &  $\nombre{479095,6}$ & $\nombre{123841,3}$\\ \hline
        \end{tabular}
      \end{center}
    \item La méthode BF permet de stabiliser la provision par rapport à
      la méthode Chain-Ladder, en particulier pour les périodes
      récentes.
    \end{enumerate}
  \end{sol}
\end{exercice}

\begin{exercice}
  En utilisant le triangle des paiements cumulatifs suivants
  \begin{center}
    \begin{tabular}{|c|c c c c c c c|}\hline
      Année & $1$ & $2$ & $3$ & $4$ & $5$ & $6$ & $7$\\ \hline
      1993 & $\nombre{1780}$ & $\nombre{2673}$ & $\nombre{2874}$ & $\nombre{3094}$ & $\nombre{3157}$ & $\nombre{3166}$ & $\nombre{3166}$ \\
      1994 & $\nombre{3226}$ & $\nombre{4219}$ & $\nombre{4532}$ & $\nombre{4881}$ & $\nombre{5144}$ & $\nombre{5199}$ & \\
      1995 & $\nombre{3652}$ & $\nombre{4989}$ & $\nombre{5762}$ & $\nombre{6436}$ & $\nombre{6720}$ & & \\
      1996 & $\nombre{2723}$ & $\nombre{4301}$ & $\nombre{5526}$ & $\nombre{6231}$ & & & \\
      1997 & $\nombre{2923}$ & $\nombre{4666}$ & $\nombre{5349}$ & & & & \\
      1998 & $\nombre{2990}$ & $\nombre{5417}$ & & & & & \\
      1999 & $\nombre{3917}$ & & & & & &\\ \hline
    \end{tabular}
  \end{center}
  répondre à la question suivante.
  \begin{enumerate}
  \item Déterminer la provision pour l'année d'accident 1998 par la
    méthode Chain-Ladder.
  \item Déterminer la provision pour les années d'accident 1998 par la
    méthode London Chain.
  \item Pourquoi peut-on dire que la méthode London Chain inclut la
    méthode Chain Ladder?
  \end{enumerate}
  \begin{rep}
    \begin{enumerate}
    \item $\nombre{1829}$
    \item $\nombre{1854,575}$
    \end{enumerate}
  \end{rep}
  \begin{sol}
    \begin{enumerate}
    \item On a
      \begin{align*}
        \lambda_1 &= \frac{\nombre{2673} + \nombre{4219} + \nombre{4989}
                    +\nombre{4301} + \nombre{4666} + \nombre{5417}}{\nombre{1780} +
                    \nombre{3226} + \nombre{3652} + \nombre{2723} + \nombre{2923} + \nombre{2990}}\\
                  &=  \frac{\nombre{26265}}{\nombre{17294}}\\
                  &= 1,518735\\
        \lambda_2 &=  1,153252\\
        \lambda_3 &= 1,104205\\
        \lambda_4 &=  1,042329\\
        \lambda_5 &= 1,007710\\
        \lambda_6 &= 1,000000.
      \end{align*}
      Le triangle complété est
      \begin{center}
        \begin{tabular}{|l|l l l l l l l|}\hline
          Année & $1$ & $2$ & $3$ & $4$ & $5$ & $6$ & $7$\\ \hline
          1993& $\nombre{1780}$& $\nombre{2673}$& $\nombre{2874}$& $\nombre{3094}$& $\nombre{3157}$& $\nombre{3166}$& $\nombre{3166}$ \\
          1994& $\nombre{3226}$& $\nombre{4219}$& $\nombre{4532}$& $\nombre{4881}$& $\nombre{5144}$& $\nombre{5199}$& $\nombre{5199}$ \\
          1995& $\nombre{3652}$& $\nombre{4989}$& $\nombre{5762}$& $\nombre{6436}$& $\nombre{6720}$& $\nombre{6772}$& $\nombre{6772}$ \\
          1996& $\nombre{2723}$& $\nombre{4301}$& $\nombre{5526}$& $\nombre{6231}$& $\nombre{6495}$& $\nombre{6545}$& $\nombre{6545}$ \\
          1997& $\nombre{2923}$& $\nombre{4666}$& $\nombre{5349}$& $\nombre{5906}$& $\nombre{6156}$& $\nombre{6204}$& $\nombre{6204}$ \\
          1998& $\nombre{2990}$& $\nombre{5417}$& $\nombre{6247}$& $\nombre{6898}$& $\nombre{7190}$& $\nombre{7246}$& $\nombre{7246}$ \\
          1999& $\nombre{3917}$& $\nombre{5949}$& $\nombre{6861}$& $\nombre{7575}$& $\nombre{7896}$& $\nombre{7957}$& $\nombre{7957}$\\\hline
        \end{tabular}
      \end{center}
      Pour 1998, le montant ultime payé est $\nombre{7246}$, et le
      montant de la provision est la différence entre le montant ultime
      et le montant total payé:
      $\nombre{7246} - \nombre{5417} = \nombre{1829}$.
    \item On a
      \begin{align*}
        \overline{C}_k^{(k)} &= \frac{1}{n - k}\sum_{i=1}^{n-k}C_{i,k}\\
        \overline{C}_1^{(1)} &=
                               \frac{(\nombre{1780}+\nombre{3226}+\nombre{3652}+\nombre{2723}+\nombre{2923}+\nombre{2990})}{7-1}\\
                             &= \nombre{2882,33} \\
        \overline{C}_2^{(2)} &= \nombre{4169,6} \\
        \overline{C}_3^{(3)} &= \nombre{4673,5} \\
        \overline{C}_4^{(4)} &= \nombre{4803,67} \\
        \overline{C}_5^{(5)} &= \nombre{4150,5} \\
        \overline{C}_6^{(6)} &= \nombre{3166}
      \end{align*}
      \begin{align*}
        \overline{C}_{k+1}^{(k)} &= \frac{1}{n-k} \sum_{i=1}^{n-k}
                                   C_{i,k+1}\\
        \overline{C}_2^{(1)} &=
                               \frac{(\nombre{2673}+\nombre{4219}+\nombre{4989}+\nombre{4301}+\nombre{4666}+\nombre{5417})}{7-1}\\
                                 &= \nombre{4377,5} \\
        \overline{C}_3^{(2)} &= \nombre{4808,6} \\
        \overline{C}_4^{(3)} &= \nombre{5160,5} \\
        \overline{C}_5^{(4)} &= \nombre{5007} \\
        \overline{C}_6^{(5)} &= \nombre{4182,5} \\
        \overline{C}_7^{(6)} &= \nombre{3166}
      \end{align*}
      \begin{align*}
        \hat{\lambda}_1 &= \frac{\frac{1}{7-1} \sum_{i=1}^{7-1} C_{i,1} C_{i,2} - \overline{C}_{2}^{(1)} \overline{C}_{1}^{(1)}}
                          {\frac{1}{7-1} \sum_{i=1}^{7-1}  C_{i,1}^2 - \overline{C}_{1}^{(1)2} }   \\
                        &=   \frac{\frac{1}{6}(\nombre{1780}*\nombre{2673} +
                          \nombre{3226}*\nombre{4219} + \ldots + \nombre{2990}*\nombre{5417})  - \nombre{2882,33}*\nombre{4377,5}}
                          {\frac{1}{6} (\nombre{1780}^2+ \nombre{3226}^2 + \nombre{3652}^2 + \nombre{2723}^2+ \nombre{2923}^2+ \nombre{2990}^2) - \nombre{2882,33}^2}\\
                        &= \frac{\nombre{13022572} -
                          \nombre{12617400}}{\nombre{8635223} - \nombre{7965547}}\\
                        &= 0,6050274\\
        \hat{\lambda}_2 &= 1,266904\\
        \hat{\lambda}_3 &= 1,172035\\
                        &\ldots
      \end{align*}
      \begin{align*}
        \hat{\alpha}_1 &= \overline{C}_{2}^{(1)} -
                         \hat{\lambda}_1\overline{C}_1^{(1)}\\
                       &= \nombre{4377,5} - (0,6050274)(\nombre{2882,33})\\
                       &= \nombre{2633,611} \\
        \hat{\alpha}_2 &= -\nombre{473,8829} \\
        \hat{\alpha}_3 &= -\nombre{317,0056} \\
                       &\ldots
      \end{align*}
      On peut alors calculer la provision demandée:
      \begin{align*}
        C_{6,3} &= \hat{\lambda}_2C_{6,2} + \hat{\alpha}_2\\
                &= (1,266904)(\nombre{5417}) -473,8829 \\
                &= \nombre{6388,936} \\
        C_{6,4} &= \nombre{7171,051} \\
                &\ldots
      \end{align*}
    \item Si les termes $\alpha$ sont égaux à $0$, on peut retrouver
      les résultats du modèle Chain-Ladder.
    \end{enumerate}
  \end{sol}
\end{exercice}

\Closesolutionfile{solutions}
\Closesolutionfile{reponses}

%%% Local Variables:
%%% mode: latex
%%% TeX-master: "provisionnement-assurance-iard"
%%% TeX-engine: xetex
%%% coding: utf-8
%%% End:

\include{stochastique}

\appendix
%%% Copyright (C) 2019 Vincent Goulet, Frédérick Guillot, Mathieu Pigeon
%%%
%%% Ce fichier fait partie du projet
%%% «Provisionnement en assurance IARD»
%%% https://gitlab.com/vigou3/provisionnement-assurance-iard
%%%
%%% Cette création est mise à disposition sous licence
%%% Attribution-Partage dans les mêmes conditions 4.0
%%% International de Creative Commons.
%%% https://creativecommons.org/licenses/by-sa/4.0/

\chapter{Solutions}
\label{chap:solutions}

\input{solutions-presentation}
\input{solutions-deterministe}
\input{solutions-stochastique}

%%% Local Variables:
%%% TeX-master: "provisionnement-assurance-iard"
%%% TeX-engine: xetex
%%% coding: utf-8
%%% End:


\bibliography{provisionnement}

\cleardoublepage
\printindex

\pagestyle{empty}

\cleartoverso
%%% Copyright (C) 2018 Vincent Goulet
%%%
%%% Ce fichier fait partie du projet
%%% «Provisionnement en assurance IARD»
%%% http://github.com/vigou3/provisionnement-assurance-iard
%%%
%%% Cette création est mise à disposition selon le contrat
%%% Attribution-Partage dans les mêmes conditions 4.0
%%% International de Creative Commons.
%%% http://creativecommons.org/licenses/by-sa/4.0/

\vspace*{\fill}

\begingroup
\calccentering{\unitlength}
\begin{adjustwidth*}{\unitlength}{-\unitlength}
  \begin{flushleft}
    \small %
    Ce document a été produit avec le système de mise en page
    {\XeLaTeX}. Le texte principal est en Lucida Bright~OT 11~points,
    les mathématiques en Lucida Bright Math~OT, le code informatique
    en Lucida Grande Mono~DK et les titres en Adobe Myriad~Pro. Des
    icônes proviennent de la police Font~Awesome. Les graphiques ont
    été réalisés avec R.
  \end{flushleft}
\end{adjustwidth*}
\endgroup
\vfill


\cleartoverso
%%% Copyright (C) 2017 Vincent Goulet
%%%
%%% Ce fichier fait partie du projet
%%% «Provisionnement en assurance IARD»
%%% http://github.com/vigou3/provisionnement-assurance-iard
%%%
%%% Cette création est mise à disposition selon le contrat
%%% Attribution-Partage dans les mêmes conditions 4.0
%%% International de Creative Commons.
%%% http://creativecommons.org/licenses/by-sa/4.0/

\begingroup

\TPGrid{3}{36}
\textblockorigin{0mm}{0mm}
\setlength{\parindent}{0mm}
\setlength{\banderougewidth}{2\TPHorizModule}
\setlength{\bandeorwidth}{\TPHorizModule}
\setlength{\gapwidth}{2pt}
\addtolength{\bandeorwidth}{-\gapwidth}

%% bandeau identitaire arrière
\begin{textblock*}{8.5in}[0,1](0mm,30\TPVertModule)
  \textcolor{or}{\rule{\bandeorwidth}{\TPVertModule}}%      % bande or
  \rule{\gapwidth}{0pt}%                                    % filet
  \textcolor{rouge}{\rule{\banderougewidth}{\TPVertModule}} % bande rouge
\end{textblock*}

% code-barre
\begin{textblock*}{0.9\TPHorizModule}(0.1\TPHorizModule,25\TPVertModule)
  % \includegraphics[height=4\TPVertModule]{codebarre_\ISBN}
\end{textblock*}

\endgroup

%%% Local Variables:
%%% mode: latex
%%% TeX-engine: xetex
%%% TeX-master: "provisionnement-assurance-iard"
%%% coding: utf-8
%%% End:


\end{document}

%%% Local Variables:
%%% TeX-master: t
%%% TeX-engine: xetex
%%% coding: utf-8
%%% End:
